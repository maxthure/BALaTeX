\chapter{Introduction}
\label{ch:introduction}
This thesis deals with Bounded History Encodings \cite{chomicki1995efficient} and their experimental application on an online automotive marketplace.

The theoretical framework for this thesis is set by \cite{borgwardt2015temporalizing}.
In \cite{borgwardt2015temporalizing}, a temporal version of ontology-based data access (OBDA) is considered.
A generic temporal query language is presented.
The reasoning task of the temporal OBDA is reduced to query answering over temporal databases,
the so-called temporal database monitoring problem.
Three approaches to solve this problem are presented, including an algorithm that guarantees a BHE.
This work builds upon the latter approach.

This thesis ties in with \cite{borgwardt2015temporalizing} and checks the applicability of the algorithm by means of a
concrete application on publicly available data, collected from an online automotive marketplace.
The investigation carried out is structured in four steps.
REF Chapter 1 shows how the publicly accessible data is regularly extracted from an online portal into a database.
In REF Chapter 2 an attempt is made to formulate practical and meaningful queries in the language of \cite{borgwardt2015temporalizing}.
It is to be checked whether the query language should be extended by additional operators.
Supplementary to practical and meaningful queries, random queries are formulated based on the idea from \cite{schneider2019formally}.
This procedure ensures that the results are valid for the entire language from \cite{borgwardt2015temporalizing}.
The results of the second chapter are used in the third chapter to construct concrete extensions of the already existent query language.
To analyze the algorithm, in the fourth chapter the extended language is applied to the extracted data from the online automotive markets.
The goal of this chapter is to evaluate how well the queries can be answered with the help of a BHE.
Criteria for this are the size of the encoding, the time to answer one query per time and the usefulness of the answers.
On the basis of the above-mentioned investigations, the final REF chapter 5 summarizes the results in order to answer the research question,
to which extent a BHE is helpful in a practical application.

%! Author = thure
%! Date = 09.05.20

% Preamble
\documentclass[11pt]{article}

% Packages
\usepackage{amsmath}
\usepackage{amsthm}
\usepackage{wasysym}
\usepackage{booktabs}
\usepackage{float}
\usepackage{amsfonts}

% Styles
\newtheorem{theorem}{Theorem}[section]
\newtheorem{corollary}[theorem]{Corollary}
\newtheorem{lemma}[theorem]{Lemma}
\newtheorem{proposition}[theorem]{Proposition}
\theoremstyle{definition}
\newtheorem{definition}[theorem]{Definition}

% Info
\title{Adding Operators. ACHTUNG NICHT FERTIG!}
\author{Thure Nebendahl}

% Document
\begin{document}
    \maketitle

    Third, the algorithm is extended by numerical predicates.
    For $\Circle \phi$, $\CIRCLE \phi$, $\Circle^{-} \phi$ and $\CIRCLE^{-} \phi$ the numerical predicates simplify composition, e.g. $\Circle_{3} \phi_{1} = \Circle (\Circle (\Circle \phi))$.
    For $\Box \phi$, $\Box^{-} \phi$, $\Diamond \phi$, $\Diamond^{-} \phi$, $\phi_{1} \mathsf{U} \phi_{2}$, and $\phi_{1} \mathsf{S} \phi_{2}$ numerical predicates restrict the number of time points,
    e.g. $\Box_{5} \phi$ (for 5 time points) or $\Diamond_{5} \phi$ (some time in 5 time points).

    The semantics of TQs from DEFINITION XY are extended as follows:

    \begin{definition}[semantics of temporal queries with numerical predicates]
        Let $\phi$ be a TQ, $\mathfrak{I} = (I_{i})_{0 \leq i \leq n}$ a sequence of interpretations over a common domain,
        $\mathfrak{a}:\mathsf{FVar}(\phi) \rightarrow \mathsf{N}_{\mathsf{C}}$ a variable assignment, $i$ be an integer with $0 \leq i \leq n$ and $p$ be an integer with $p > 0$.
        The $satisfaction\ relation\ \mathfrak{I}, i \models \mathfrak{a}(\phi)$ is defined by induction on the structure of $\phi$ as follows:
        \begin{table}[H]
            \centering
            \begin{tabular*}{\textwidth}{@{}ll@{}}
                \toprule
                \phi                            & $\mathfrak{I}, i \models \mathfrak{a}(\phi)$                                                        \\ \midrule
                \Circle_{p} \phi_{1}            & $i+p \leq n$ and $\mathfrak{I},i+p \models \mathfrak{a}(\phi_{1})$                                    \\
                \CIRCLE_{p} \phi_{1}            & $i+p \leq n$ implies $\mathfrak{I},i+p \models \mathfrak{a}(\phi_{1})$                                \\
                \Circle^{-}_{p} \phi_{1}        & $i-p \geq 0$ and $\mathfrak{I},i-p \models \mathfrak{a}(\phi_{1})$                                    \\
                \CIRCLE^{-}_{p} \phi_{1}        & $i-p \geq 0$ implies $\mathfrak{I},i-p \models \mathfrak{a}(\phi_{1})$                                \\
                \Box_{p} \phi_{1}               & $\mathfrak{I}, k \models \mathfrak{a}(\phi_{1})$ for all $k$, $i \leq k \leq i+p$                     \\
                \Box^{-}_{p} \phi_{1}           & $\mathfrak{I}, k \models \mathfrak{a}(\phi_{1})$ for all $k$, $i-p \leq k \leq i$                     \\
                \Diamond_{p} \phi_{1}           & $\mathfrak{I}, k \models \mathfrak{a}(\phi_{1})$ for some $k$, $i \leq k \leq i+p$                    \\
                \Diamond^{-}_{p} \phi_{1}       & $\mathfrak{I}, k \models \mathfrak{a}(\phi_{1})$ for some $k$, $i-p \leq k \leq i$                    \\
                \phi_{1} \mathsf{U}_{p} \phi_{2}& there is $k$, $i \leq k \leq i+p$, with $\mathfrak{I}, k \models \mathfrak{a}_{\phi_{2}}(\phi_{2})$   \\
                                                & and $\mathfrak{I}, j \models \mathfrak{a}_{\phi_{1}}(\phi_{1})$ for all $j, i \leq j < k$             \\
                \phi_{1} \mathsf{S}_{p} \phi_{2}& there is $k$, $i-p \leq k \leq i$, with $\mathfrak{I}, k \models \mathfrak{a}_{\phi_{2}}(\phi_{2})$   \\
                                                & and $\mathfrak{I}, j \models \mathfrak{a}_{\phi_{1}}(\phi_{1})$ for all $j, k < j \leq i$             \\ \bottomrule
            \end{tabular*}
            \caption{semantics of TQs with numerical predicates}
            \label{tab:tqsemanticswnp}
        \end{table}
    \end{definition}

    There are again equivalences similar to PROPOSITION XY.

    \begin{proposition}
        \label{prop1}
        For $\mathfrak{a}:\mathsf{FVar}(\phi) \rightarrow \mathsf{N}_{\mathsf{C}}$ and $0 \leq i \leq n$, we have
        \begin{enumerate}
            \item $\mathfrak{I}, i \models \mathfrak{a}(\Box_{p} \phi_{1})$ iff
            \begin{itemize}
                \item $\mathfrak{I}, i \models \mathfrak{a}(\phi_{1})$ and
                \item $i < n$ implies $\mathfrak{I}, i+1 \models \mathfrak{a}(\Box_{p-1} \phi_{1})$
            \end{itemize}
            \item $\mathfrak{I}, i \models \mathfrak{a}(\Box^{-}_{p} \phi_{1})$ iff
            \begin{itemize}
                \item $\mathfrak{I}, i \models \mathfrak{a}(\phi_{1})$ and
                \item $i > 0$ implies $\mathfrak{I}, i-1 \models \mathfrak{a}(\Box^{-}_{p-1} \phi_{1})$
            \end{itemize}
            \item $\mathfrak{I}, i \models \mathfrak{a}(\Diamond_{p} \phi_{1})$ iff
            \begin{itemize}
                \item $\mathfrak{I}, i \models \mathfrak{a}(\phi_{1})$ or
                \item $i < n$ and $\mathfrak{I}, i+1 \models \mathfrak{a}(\Diamond_{p-1} \phi_{1})$
            \end{itemize}
            \item $\mathfrak{I}, i \models \mathfrak{a}(\Diamond^{-}_{p} \phi_{1})$ iff
            \begin{itemize}
                \item $\mathfrak{I}, i \models \mathfrak{a}(\phi_{1})$ or
                \item $i > 0$ and $\mathfrak{I}, i-1 \models \mathfrak{a}(\Diamond^{-}_{p-1} \phi_{1})$
            \end{itemize}
            \item $\mathfrak{I}, i \models \mathfrak{a}(\phi_{1} \mathsf{U}_{p} \phi_{2})$ iff
            \begin{itemize}
                \item $\mathfrak{I}, i \models \mathfrak{a}_{\phi_{2}}(\phi_{2})$ or
                \item $\mathfrak{I}, i \models \mathfrak{a}_{\phi_{1}}(\phi_{1})$ and $\mathfrak{I}, i+1 \models \mathfrak{a}(\phi_{1} \mathsf{U}_{p-1} \phi_{2})$
            \end{itemize}
            \item $\mathfrak{I}, i \models \mathfrak{a}(\phi_{1} \mathsf{S}_{p} \phi_{2})$ iff
            \begin{itemize}
                \item $\mathfrak{I}, k \models \mathfrak{a}_{\phi_{2}}(\phi_{2})$ or
                \item $\mathfrak{I}, i \models \mathfrak{a}_{\phi_{1}}(\phi_{1})$ and $\mathfrak{I}, i-1 \models \mathfrak{a}(\phi_{1} \mathsf{S}_{p-1} \phi_{2})$
            \end{itemize}
        \end{enumerate}
    \end{proposition}




    \bibliography{main}
    \bibliographystyle{plain}

\end{document}
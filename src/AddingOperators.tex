%! Author = thure
%! Date = 09.05.20

% Preamble
\documentclass[11pt]{article}

% Packages
\usepackage{amsmath}
\usepackage{amsthm}
\usepackage{wasysym}
\usepackage{booktabs}
\usepackage{float}
\usepackage{amsfonts}

% Styles
\newtheorem{theorem}{Theorem}[section]
\newtheorem{corollary}[theorem]{Corollary}
\newtheorem{lemma}[theorem]{Lemma}
\newtheorem{proposition}[theorem]{Proposition}
\theoremstyle{definition}
\newtheorem{definition}[theorem]{Definition}

% Info
\title{Adding Missing Operators (APPENDIX)}
\author{Thure Nebendahl}

% Document
\begin{document}
    \maketitle

    \begin{proof}
        \begin{enumerate}
            \item $\Box \phi_{1} \equiv \phi_{1} \wedge \CIRCLE \Box \phi_{1}$
            \begin{align}
                &\mathfrak{I}, i \models \mathfrak{a}(\Box \phi_{1}) \label{proof1.1.1} \\
                \Leftrightarrow &\mathfrak{I}, k \models \mathfrak{a}(\phi_{1}) \text{ for all } k, i \leq k \leq n \label{proof1.1.2} \\
                \Leftrightarrow &\mathfrak{I}, i \models \mathfrak{a}(\phi_{1}) \text{ and } (i < n \text{ implies } \label{proof1.1.3} \\
                &\mathfrak{I}, k \models \mathfrak{a}(\phi_{1}) \text{ for all } k, i+1 \leq k \leq n) \nonumber \\
                \Leftrightarrow &\mathfrak{I}, i \models \mathfrak{a}(\phi_{1}) \text{ and } (i < n \text{ implies } \mathfrak{I}, i+1 \models \mathfrak{a}(\Box\phi_{1})) \label{proof1.1.4} \\
                \Leftrightarrow &\mathfrak{I}, i \models \mathfrak{a}(\phi_{1} \wedge \CIRCLE \Box \phi_{1}) \label{proof1.1.5}
            \end{align}
            \eqref{proof1.1.3} is equivalent to \eqref{proof1.1.2} because
            \begin{itemize}
                \item in case $i < n$, the query needs to be satisfied now, at time point $i$, and at all future time points $k$, $i+1 \leq k \leq n$, in order to be satisfied.
                Since $i < n$ is true, the satisfaction of future time points depends solely on the second part of the "implies"-statement; and
                \item in case $i = n$, the query needs to be satisfied now, at time point $i$, in order to be satisfied.
                There are no future time points $k$, $i+1 \leq k \leq n$.
                Since $i = n$ is true, $\mathfrak{I}, i \models \mathfrak{a}(\phi_{1})$ is equivalent to $\mathfrak{I}, k \models \mathfrak{a}(\phi_{1})$ for all $k$, $n \leq k \leq n$,
                and $i < n$ is not true, thus the "implies"-statement has no effect on satisfaction.
            \end{itemize}

            \item $\Box^{-} \phi_{1} \equiv \phi_{1} \wedge \CIRCLE^{-} \Box^{-} \phi_{1}$
            \begin{align}
                &\mathfrak{I}, i \models \mathfrak{a}(\Box^{-} \phi_{1}) \label{proof1.2.1} \\
                \Leftrightarrow &\mathfrak{I}, k \models \mathfrak{a}(\phi_{1}) \text{ for all } k, 0 \leq k \leq i \label{proof1.2.2} \\
                \Leftrightarrow &\mathfrak{I}, i \models \mathfrak{a}(\phi_{1}) \text{ and } (i > 0 \text{ implies } \label{proof1.2.3} \\
                &\mathfrak{I}, k \models \mathfrak{a}(\phi_{1}) \text{ for all } k, 0 \leq k \leq i-1) \nonumber \\
                \Leftrightarrow &\mathfrak{I}, i \models \mathfrak{a}(\phi_{1}) \text{ and } (i > 0 \text{ implies } \mathfrak{I}, i-1 \models \mathfrak{a}(\Box^{-}\phi_{1})) \label{proof1.2.4} \\
                \Leftrightarrow &\mathfrak{I}, i \models \mathfrak{a}(\phi_{1} \wedge \CIRCLE^{-} \Box^{-} \phi_{1}) \label{proof1.2.5}
            \end{align}
            \eqref{proof1.2.3} is equivalent to \eqref{proof1.2.2} because
            \begin{itemize}
                \item in case $i > 0$, the query needs to be satisfied now, at time point $i$, and at all past time points $k$, $0 \leq k \leq i-1$, in order to be satisfied.
                Since $i > 0$ is true, the satisfaction of past time points depends solely on the second part of the "implies"-statement; and
                \item in case $i = 0$, the query needs to be satisfied now, at time point $i$, in order to be satisfied.
                There are no past time points $k$, $0 \leq k \leq i-1$.
                Since $i = 0$ is true, $\mathfrak{I}, i \models \mathfrak{a}(\phi_{1})$ is equivalent to $\mathfrak{I}, k \models \mathfrak{a}(\phi_{1})$ for all $k$, $0 \leq k \leq 0$,
                and $i > 0$ is not true, thus the "implies"-statement has no effect on satisfaction.
            \end{itemize}

            \item $\Diamond \phi_{1} \equiv \phi_{1} \vee \Circle \Diamond \phi_{1}$
            \begin{align}
                &\mathfrak{I}, i \models \mathfrak{a}(\Diamond \phi_{1}) \label{proof1.3.1} \\
                \Leftrightarrow &\mathfrak{I}, k \models \mathfrak{a}(\phi_{1}) \text{ for some } k, i \leq k \leq n \label{proof1.3.2} \\
                \Leftrightarrow &\mathfrak{I}, i \models \mathfrak{a}(\phi_{1}) \text{ or } (i < n \text{ and } \label{proof1.3.3} \\
                &\mathfrak{I}, k \models \mathfrak{a}(\phi_{1}) \text{ for some } k, i+1 \leq k \leq n) \nonumber \\
                \Leftrightarrow &\mathfrak{I}, i \models \mathfrak{a}(\phi_{1}) \text{ or } (i < n \text{ and } \mathfrak{I}, i+1 \models \mathfrak{a}(\Diamond \phi_{1})) \label{proof1.3.4} \\
                \Leftrightarrow &\mathfrak{I}, i \models \mathfrak{a}(\phi_{1} \vee \Circle \Diamond \phi_{1}) \label{proof1.3.5}
            \end{align}
            \eqref{proof1.3.3} is equivalent to \eqref{proof1.3.2} because
            \begin{itemize}
                \item in case $i < n$, the query needs to be satisfied now, at time point $i$, or at any future time point $k$, $i+1 \leq k \leq n$, in order to be satisfied.
                Since $i < n$ is true, the satisfaction of future time points depends solely on the second part of the "and"-statement; and
                \item in case $i = n$, the query needs to be satisfied now, at time point $i$, in order to be satisfied.
                There are no future time points $k$, $i+1 \leq k \leq n$.
                Since $i = n$ is true, $\mathfrak{I}, i \models \mathfrak{a}(\phi_{1})$ is equivalent to $\mathfrak{I}, k \models \mathfrak{a}(\phi_{1})$ for some $k$, $n \leq k \leq n$,
                and $i > 0$ is not true, thus the "and"-statement has no effect on satisfaction.
            \end{itemize}

            \item $\Diamond^{-} \phi_{1} \equiv \phi_{1} \vee \Circle^{-} \Diamond^{-} \phi_{1}$
            \begin{align}
                &\mathfrak{I}, i \models \mathfrak{a}(\Diamond^{-} \phi_{1}) \label{proof1.4.1} \\
                \Leftrightarrow &\mathfrak{I}, k \models \mathfrak{a}(\phi_{1}) \text{ for some } k, 0 \leq k \leq i \label{proof1.4.2} \\
                \Leftrightarrow &\mathfrak{I}, i \models \mathfrak{a}(\phi_{1}) \text{ or } (i > 0 \text{ and } \label{proof1.4.3} \\
                &\mathfrak{I}, k \models \mathfrak{a}(\phi_{1}) \text{ for some } k, 0 \leq k \leq i-1) \nonumber \\
                \Leftrightarrow &\mathfrak{I}, i \models \mathfrak{a}(\phi_{1}) \text{ or } (i > 0 \text{ and } \mathfrak{I}, i-1 \models \mathfrak{a}(\Diamond^{-} \phi_{1})) \label{proof1.4.4} \\
                \Leftrightarrow &\mathfrak{I}, i \models \mathfrak{a}(\phi_{1} \vee \Circle^{-} \Diamond^{-} \phi_{1}) \label{proof1.4.5}
            \end{align}
            \eqref{proof1.4.3} is equivalent to \eqref{proof1.4.2} because
            \begin{itemize}
                \item in case $i > 0$, the query needs to be satisfied now, at time point $i$, or at any past time point $k$, $0 \leq k \leq i-1$, in order to be satisfied.
                Since $i > 0$ is true, the satisfaction of past time points depends solely on the second part of the "and"-statement; and
                \item in case $i = 0$, the query needs to be satisfied now, at time point $i$, in order to be satisfied.
                There are no past time points $k$, $0 \leq k \leq i-1$.
                Since $i = 0$ is true, $\mathfrak{I}, i \models \mathfrak{a}(\phi_{1})$ is equivalent to $\mathfrak{I}, k \models \mathfrak{a}(\phi_{1})$ for some $k$, $0 \leq k \leq 0$,
                and $i > 0$ is not true, thus the "and"-statement has no effect on satisfaction.
            \end{itemize}
        \end{enumerate}
    \end{proof}

    \bibliography{main}
    \bibliographystyle{plain}

\end{document}
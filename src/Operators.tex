\chapter{Extension of the BHE ACHTUNG NICHT FERTIG}
\label{ch:extension}
It became clear that both the language from \cite{borgwardt2015temporalizing} and consequently the implementation of the bounded history encoding from \cite{borgwardt2015temporalizing}
must be extended in order to provide relevant answers to the PTQs specified in CHAPTER REF CHAPTER 2.
In this chapter, the extensions are introduced one by one and EXAMINED FOR FEASIBILITY.

\section{Extensions}
\label{sec:extension/extensions}

\subsection{Operators from \cite{borgwardt2015temporalizing} not yet implemented}
\label{subsec:extension/extensions/operatorsnotimplemented}
First, the algorithm presented in \cite{borgwardt2015temporalizing} is extended by the operators
$\Box \phi$ (always), $\Box^{-} \phi$ (always in the past), $\Diamond \phi$ (eventually), $\Diamond^{-} \phi$ (some time in the past).
Therefore, the definitions of $\mathsf{eval}^{n}(\alpha)$, $\Phi_{0}(\psi)$ and $\Phi_{i}(\psi)$ from \cite{borgwardt2015temporalizing} are modified,
upon which the modifications are then proven to be correct.
As a reminder, the semantics of these four TQs are defined as follows:
\begin{definition}[semantics of temporal queries ct.\ Definition 3.3 in \cite{borgwardt2015temporalizing}]
    \label{def:Extension/operatorsnotimplemented/semantics}
    Let $\phi$ be a TQ, $\mathfrak{I} = (I_{i})_{0 \leq i \leq n}$ a sequence of interpretations over a common domain,
    $\mathfrak{a}:\mathsf{FVar}(\phi) \rightarrow \mathsf{N}_{\mathsf{C}}$ a variable assignment, and $i$ be an integer with $0 \leq i \leq n$.
    The $satisfaction\ relation\ \mathfrak{I}, i \models \mathfrak{a}(\phi)$ is defined by induction on the structure of $\phi$ as follows:
    \begin{table}[H]
        \centering
        \begin{tabular*}{\textwidth}{@{}ll@{}}
            \toprule
            \phi                    & $\mathfrak{I}, i \models \mathfrak{a}(\phi)$ iff                                   \\ \midrule
            \Box \phi_{1}           & $\mathfrak{I}, k \models \mathfrak{a}(\phi_{1})$ for all $k, i \leq k \leq n$ \\
            \Box^{-} \phi_{1}       & $\mathfrak{I}, k \models \mathfrak{a}(\phi_{1})$ for all $k, 0 \leq k \leq i$ \\
            \Diamond \phi_{1}       & $\mathfrak{I}, k \models \mathfrak{a}(\phi_{1})$ for some $k, i \leq k \leq n$\\
            \Diamond^{-} \phi_{1}   & $\mathfrak{I}, k \models \mathfrak{a}(\phi_{1})$ for some $k, 0 \leq k \leq i$\\ \bottomrule
        \end{tabular*}
        \caption{satisfaction relation of TQs $\Box, \Box^{-}, \Diamond$ and $\Diamond^{-}$}
        \label{tab:extension/operatorsnotimplemented/satisfactionrelation}
    \end{table}
    $\mathsf{FVar}(\phi)$ denotes the set of $free\ variables$ of a TQ and is defined as the union of the sets $\mathsf{FVar}(\psi)$ of
    all queries $\psi$ occurring in $\phi$. $\mathsf{N}_{\mathsf{C}}$ denotes a set of $constants$.
    If $\mathfrak{I}, i \models \mathfrak{a}(\phi)$, then $\mathfrak{a}$ is called an $answer$ to $\phi$ w.r.t. $\mathfrak{I}$ at time point $i$.
    The set of all answers to $\phi$ w.r.t $\mathfrak{I}$ at time point $i$ is denoted by $\mathsf{Ans}(\phi, \mathfrak{I},i)$.
\end{definition}

As in \cite{borgwardt2015temporalizing}, it can be shown that
\begin{itemize}
    \item $\Box \phi_{1}$ is equivalent to $\phi_{1} \wedge \CIRCLE \Box \phi_{1}$ ;
    \item $\Diamond \phi_{1}$ is equivalent to $\phi_{1} \vee \Circle \Diamond \phi_{1}$ .
\end{itemize}
Because of the way $\Box$ is defined in \cite{borgwardt2015temporalizing}, $\CIRCLE$ has to be used instead of
$\Circle$ with the only difference that $\CIRCLE$ is tautological at the last time point.
It can similarly be shown that
\begin{itemize}
    \item $\Box^{-} \phi_{1}$ is equivalent to $\phi_{1} \wedge \CIRCLE^{-} \Box^{-} \phi_{1}$ ;
    \item $\Diamond^{-} \phi_{1}$ is equivalent to $\phi_{1} \vee \Circle^{-} \Diamond^{-} \phi_{1}$ .
\end{itemize}
Analogously to the reason as mentioned above, $\CIRCLE^{-}$ has to be used instead of $\Circle^{-}$
with the only difference that $\CIRCLE^{-}$ is tautological at the first time point. \\
Thus, at the last time point
\begin{itemize}
    \item $\Box \phi_{1}$ is equivalent to $\phi_{1}$ because $\CIRCLE \Box \phi_{1}$ is tautological
    \item $\Diamond \phi_{1}$ is equivalent to $\phi_{1}$ because $\Circle \Diamond \phi_{1}$ does not have any answers
\end{itemize}
and at the first time point
\begin{itemize}
    \item $\Box^{-} \phi_{1}$ is equivalent to $\phi_{1}$ because $\CIRCLE^{-} \Box^{-} \phi_{1}$ is tautological
    \item $\Diamond^{-} \phi_{1}$ is equivalent to $\phi_{1}$ because $\Circle^{-} \Diamond^{-} \phi_{1}$ does not have any answers
\end{itemize}

\begin{proposition}[ct.\ Proposition 3.4 in \cite{borgwardt2015temporalizing}]
    \label{prop:extension/operatorsnotimplemented/prop2}
    For $\mathfrak{a}:\mathsf{FVar}(\phi) \rightarrow \mathsf{N}_{\mathsf{C}}$ and $0 \leq i \leq n$,
    \begin{enumerate}
        \item $\mathfrak{I}, i \models \mathfrak{a}(\Box \phi_{1})$ iff
        \begin{itemize}
            \item $\mathfrak{I}, i \models \mathfrak{a}(\phi_{1})$ and
            \item $i < n$ implies $\mathfrak{I}, i+1 \models \mathfrak{a}(\Box \phi_{1})$
        \end{itemize}
        \item $\mathfrak{I}, i \models \mathfrak{a}(\Box^{-} \phi_{1})$ iff
        \begin{itemize}
            \item $\mathfrak{I}, i \models \mathfrak{a}(\phi_{1})$ and
            \item $i > 0$ implies $\mathfrak{I}, i-1 \models \mathfrak{a}(\Box^{-} \phi_{1})$
        \end{itemize}
        \item $\mathfrak{I}, i \models \mathfrak{a}(\Diamond \phi_{1})$ iff
        \begin{itemize}
            \item $\mathfrak{I}, i \models \mathfrak{a}(\phi_{1})$ or
            \item $i < n$ and $\mathfrak{I}, i+1 \models \mathfrak{a}(\Diamond \phi_{1})$
        \end{itemize}
        \item $\mathfrak{I}, i \models \mathfrak{a}(\Diamond^{-} \phi_{1})$ iff
        \begin{itemize}
            \item $\mathfrak{I}, i \models \mathfrak{a}(\phi_{1})$ or
            \item $i > 0$ and $\mathfrak{I}, i-1 \models \mathfrak{a}(\Diamond^{-} \phi_{1})$
        \end{itemize}
    \end{enumerate}
\end{proposition}

\begin{proof}
    To prove the above proposition, TWO equivalences are demonstrated here.
    The other two cases work similar AND CAN BE FOUND IN THE APPENDIX REF APPENDIX XY.
    The proof works mainly on the basis of semantics.
    \begin{enumerate}
        \item $\Box \phi_{1} \equiv \phi_{1} \wedge \CIRCLE \Box \phi_{1}$
        \begin{align}
            &\mathfrak{I}, i \models \mathfrak{a}(\Box \phi_{1}) \\
            \Leftrightarrow &\mathfrak{I}, k \models \mathfrak{a}(\phi_{1}) \text{ for all } k, i \leq k \leq n  \label{eq:extension/operatorsnotimplemented/proofProp2/eq1.2} \\
            \Leftrightarrow &\mathfrak{I}, i \models \mathfrak{a}(\phi_{1}) \text{ and } (i < n \text{ implies } \label{eq:extension/operatorsnotimplemented/proofProp2/eq1.3}\\
            &\mathfrak{I}, k \models \mathfrak{a}(\phi_{1}) \text{ for all } k, i+1 \leq k \leq n) \nonumber \\
            \Leftrightarrow &\mathfrak{I}, i \models \mathfrak{a}(\phi_{1}) \text{ and } (i < n \text{ implies } \mathfrak{I}, i+1 \models \mathfrak{a}(\Box\phi_{1})) \\
            \Leftrightarrow &\mathfrak{I}, i \models \mathfrak{a}(\phi_{1} \wedge \CIRCLE \Box \phi_{1})
        \end{align}
        REF EQ 3 is equivalent to REF EQ 2 because
        \begin{itemize}
            \item in case $i < n$, the query needs to be satisfied now, at time point $i$, and at all future time points $k$, $i+1 \leq k \leq n$, in order to be satisfied.
            Since $i < n$ is true, the satisfaction of future time points depends solely on the second part of the "implies"-statement; and
            \item in case $i = n$, the query needs to be satisfied now, at time point $i$, in order to be satisfied.
            There are no future time points $k$, $n+1 \leq k \leq n$.
            Since $i = n$ is true, $\mathfrak{I}, i \models \mathfrak{a}(\phi_{1})$ is equivalent to $\mathfrak{I}, k \models \mathfrak{a}(\phi_{1})$ for all $k$, $n \leq k \leq n$,
            and $i < n$ is not true, thus the "implies"-statement has no effect on satisfaction.
        \end{itemize}

        \item $\Box^{-} \phi_{1} \equiv \phi_{1} \wedge \CIRCLE^{-} \Box^{-} \phi_{1}$
        \begin{align}
            &\mathfrak{I}, i \models \mathfrak{a}(\Box^{-} \phi_{1}) \\
            \Leftrightarrow &\mathfrak{I}, k \models \mathfrak{a}(\phi_{1}) \text{ for all } k, 0 \leq k \leq i \label{eq:extension/operatorsnotimplemented/proofProp2/eq2.2}\\
            \Leftrightarrow &\mathfrak{I}, i \models \mathfrak{a}(\phi_{1}) \text{ and } (i > 0 \text{ implies } \label{eq:extension/operatorsnotimplemented/proofProp2/eq2.3}\\
            &\mathfrak{I}, k \models \mathfrak{a}(\phi_{1}) \text{ for all } k, 0 \leq k \leq i-1) \nonumber \\
            \Leftrightarrow &\mathfrak{I}, i \models \mathfrak{a}(\phi_{1}) \text{ and } (i > 0 \text{ implies } \mathfrak{I}, i-1 \models \mathfrak{a}(\Box^{-}\phi_{1})) \\
            \Leftrightarrow &\mathfrak{I}, i \models \mathfrak{a}(\phi_{1} \wedge \CIRCLE^{-} \Box^{-} \phi_{1})
        \end{align}
        REF EQ 3 is equivalent to REF EQ 2 because
        \begin{itemize}
            \item in case $i > 0$, the query needs to be satisfied now, at time point $i$, and at all past time points $k$, $0 \leq k \leq i-1$, in order to be satisfied.
            Since $i > 0$ is true, the satisfaction of past time points depends solely on the second part of the "implies"-statement; and
            \item in case $i = 0$, the query needs to be satisfied now, at time point $i$, in order to be satisfied.
            There are no past time points $k$, $0 \leq k \leq 0-1$.
            Since $i = 0$ is true, $\mathfrak{I}, i \models \mathfrak{a}(\phi_{1})$ is equivalent to $\mathfrak{I}, k \models \mathfrak{a}(\phi_{1})$ for all $k$, $0 \leq k \leq 0$,
            and $i > 0$ is not true, thus the "implies"-statement has no effect on satisfaction.
        \end{itemize}

        \item $\Diamond \phi_{1} \equiv \phi_{1} \vee \Circle \Diamond \phi_{1}$
        \begin{align}
            &\mathfrak{I}, i \models \mathfrak{a}(\Diamond \phi_{1}) \\
            \Leftrightarrow &\mathfrak{I}, k \models \mathfrak{a}(\phi_{1}) \text{ for some } k, i \leq k \leq n \label{eq:extension/operatorsnotimplemented/proofProp2/eq3.2}\\
            \Leftrightarrow &\mathfrak{I}, i \models \mathfrak{a}(\phi_{1}) \text{ or } (i < n \text{ and } \label{eq:extension/operatorsnotimplemented/proofProp2/eq3.3}\\
            &\mathfrak{I}, k \models \mathfrak{a}(\phi_{1}) \text{ for some } k, i+1 \leq k \leq n) \nonumber \\
            \Leftrightarrow &\mathfrak{I}, i \models \mathfrak{a}(\phi_{1}) \text{ or } (i < n \text{ and } \mathfrak{I}, i+1 \models \mathfrak{a}(\Diamond \phi_{1})) \\
            \Leftrightarrow &\mathfrak{I}, i \models \mathfrak{a}(\phi_{1} \vee \Circle \Diamond \phi_{1})
        \end{align}
        REF EQ 3 is equivalent to REF EQ 2 because
        \begin{itemize}
            \item in case $i < n$, the query needs to be satisfied now, at time point $i$, or at any future time point $k$, $i+1 \leq k \leq n$, in order to be satisfied.
            Since $i < n$ is true, the satisfaction of future time points depends solely on the second part of the "and"-statement; and
            \item in case $i = n$, the query needs to be satisfied now, at time point $i$, in order to be satisfied.
            There are no future time points $k$, $n+1 \leq k \leq n$.
            Since $i = n$ is true, $\mathfrak{I}, i \models \mathfrak{a}(\phi_{1})$ is equivalent to $\mathfrak{I}, k \models \mathfrak{a}(\phi_{1})$ for some $k$, $n \leq k \leq n$,
            and $i > 0$ is not true, thus the "and"-statement has no effect on satisfaction.
        \end{itemize}

        \item $\Diamond^{-} \phi_{1} \equiv \phi_{1} \vee \Circle^{-} \Diamond^{-} \phi_{1}$
        \begin{align}
            &\mathfrak{I}, i \models \mathfrak{a}(\Diamond^{-} \phi_{1}) \\
            \Leftrightarrow &\mathfrak{I}, k \models \mathfrak{a}(\phi_{1}) \text{ for some } k, 0 \leq k \leq i \label{eq:extension/operatorsnotimplemented/proofProp2/eq4.2}\\
            \Leftrightarrow &\mathfrak{I}, i \models \mathfrak{a}(\phi_{1}) \text{ or } (i > 0 \text{ and } \label{eq:extension/operatorsnotimplemented/proofProp2/eq4.3}\\
            &\mathfrak{I}, k \models \mathfrak{a}(\phi_{1}) \text{ for some } k, 0 \leq k \leq i-1) \nonumber \\
            \Leftrightarrow &\mathfrak{I}, i \models \mathfrak{a}(\phi_{1}) \text{ or } (i > 0 \text{ and } \mathfrak{I}, i-1 \models \mathfrak{a}(\Diamond^{-} \phi_{1})) \\
            \Leftrightarrow &\mathfrak{I}, i \models \mathfrak{a}(\phi_{1} \vee \Circle^{-} \Diamond^{-} \phi_{1})
        \end{align}
        REF EQ 3 is equivalent to REF EQ 2 because
        \begin{itemize}
            \item in case $i > 0$, the query needs to be satisfied now, at time point $i$, or at any past time point $k$, $0 \leq k \leq i-1$, in order to be satisfied.
            Since $i > 0$ is true, the satisfaction of past time points depends solely on the second part of the "and"-statement; and
            \item in case $i = 0$, the query needs to be satisfied now, at time point $i$, in order to be satisfied.
            There are no past time points $k$, $0 \leq k \leq 0-1$.
            Since $i = 0$ is true, $\mathfrak{I}, i \models \mathfrak{a}(\phi_{1})$ is equivalent to $\mathfrak{I}, k \models \mathfrak{a}(\phi_{1})$ for some $k$, $0 \leq k \leq 0$,
            and $i > 0$ is not true, thus the "and"-statement has no effect on satisfaction.
        \end{itemize}
    \end{enumerate}
\end{proof}

The semantics of the four operators can now be used to extend the algorithm specified in \cite{borgwardt2015temporalizing}.
For this the notation of $answer\ terms$ is needed.
Using the same simplification as in \cite{borgwardt2015temporalizing}, the following assumes that $\mathsf{N}_{\mathsf{V}}$, the set of $variables$, is finite
and that the answers are of the form $\mathfrak{a}:\mathsf{N}_{\mathsf{V}} \rightarrow \Delta$ instead of $\mathfrak{a}:\mathsf{FVar}(\phi) \rightarrow \Delta$.
$\mathsf{Ans}(\phi, \mathfrak{I}^{(n)})$ refers to a set of mappings $\mathfrak{a}:\mathsf{N}_{\mathsf{V}} \rightarrow \Delta$, i.e, a subset of $\Delta^{\mathsf{N}_\mathsf{V}}$.
\begin{definition}[answer\ term ct.\ Definition 6.1 in \cite{borgwardt2015temporalizing}]
    \label{def:extension/operatorsnotimplemented/answerterm}
    Let $\mathsf{FSub}(\phi)$ denote the set of all subqueries of $\phi$ of the form $\Circle \psi_{1}, \CIRCLE \psi_{1}, \Box \psi_{1}, \Diamond \psi_{1}$
    or $\psi_{1} \mathsf{U} \psi_{2}$.
    For $j \geq 0$, we denote by $\mathsf{Var}^{\phi}_{j}$ the set of all variables of the form $x^{\psi}_{j}$ for $\psi \in \mathsf{FSub}(\phi)$.
    The set $\mathsf{AT}^{i}_{\phi}$ of all $answer\ terms$ for $\phi$ at $i \geq 0$ is the smallest set satisfying the following conditions:
    \begin{itemize}
        \item Every set $A \subseteq \Delta^{\mathsf{N}_\mathsf{V}}$ is an answer term for $\phi$ at $i$.
        \item Every variable $x^{\psi}_{j} \in \mathsf{Var}^{\phi}_{j}$ with $j \leq i$ is an answer term for $\phi$ at $i$.
        \item If $\alpha_{1}$ and $\alpha_{2}$ are answer terms for $\phi$ at $i$, then so are $\alpha_{1} \cap \alpha_{2}$ and $\alpha_{1} \cup \alpha_{2}$.
    \end{itemize}
\end{definition}

The functions $\mathsf{eval}^{n}: \mathsf{AT}^{n}_{\phi} \rightarrow 2^{\Delta^{\mathsf{N}_\mathsf{V}}}, n \geq 0$ in \cite{borgwardt2015temporalizing} have then to be extended as follows:
\begin{table}[H]
    \centering
    \begin{tabular*}{\textwidth}{@{}ll@{}}
        \toprule
        \alpha                                      & $\mathsf{eval}^{n}(\alpha)$                                   \\ \midrule
        [\ldots]                & [\ldots] \\
        $x^{\Box \psi_{1}}_{j}$ with $j < n$        & $\mathsf{Ans}(\Box \psi_{1}, \mathfrak{I}^{(n)}, j+1)$        \\
        $x^{\Diamond \psi_{1}}_{j}$ with $j < n$    & $\mathsf{Ans}(\Diamond \psi_{1}, \mathfrak{I}^{(n)}, j+1)$    \\
        $x^{\Box \psi_{1}}_{n}$                     & $\Delta^{\mathsf{N}_\mathsf{V}}$                              \\
        $x^{\Diamond \psi_{1}}_{n}$                 & \emptyset                                                     \\ \bottomrule
    \end{tabular*}
    \caption{$\mathsf{eval}^{n}(\alpha)$ with $\Box, \Box^{-}, \Diamond$ and $\Diamond^{-}$}
    \label{tab:extension/operatorsnotimplemented/eval}
\end{table}

The function $\Phi_{0}(\psi): \mathsf{Sub}(\phi) \rightarrow \mathsf{AT}^{0}_{\phi}$ in \cite{borgwardt2015temporalizing} has to be extended as follows:
\begin{table}[H]
    \centering
    \begin{tabular*}{\textwidth}{@{}ll@{}}
        \toprule
        \psi                    & $\Phi_{0}(\psi)$                                    \\ \midrule
        [\ldots]                & [\ldots] \\
        \Box \psi_{1}           & $\Phi_{0}(\psi_{1}) \cap x^{\Box \psi_{1}}_{0}$     \\
        \Box^{-} \psi_{1}       & $\Phi_{0}(\psi_{1})$                                \\
        \Diamond \psi_{1}       & $\Phi_{0}(\psi_{1}) \cup x^{\Diamond \psi_{1}}_{0}$ \\
        \Diamond^{-} \psi_{1}   & $\Phi_{0}(\psi_{1})$                                \\ \bottomrule
    \end{tabular*}
    \caption{$\Phi_{0}(\psi)$ with $\Box, \Box^{-}, \Diamond$ and $\Diamond^{-}$}
    \label{tab:extension/operatorsnotimplemented/phi0}
\end{table}

The function $\Phi^{0}_{i}(\psi): \mathsf{Sub}(\phi) \rightarrow \mathsf{AT}^{i}_{\phi},\ i>0$ in \cite{borgwardt2015temporalizing} has to be extended as follows:
\begin{table}[H]
    \centering
    \begin{tabular*}{\textwidth}{@{}ll@{}}
        \toprule
        \psi                    & $\Phi^{0}_{i}(\psi)$                                            \\ \midrule
        [\ldots]                & [\ldots] \\
        \Box \psi_{1}           & $\Phi^{0}_{i}(\psi_{1}) \cap x^{\Box \psi_{1}}_{i}$             \\
        \Box^{-} \psi_{1}       & $\Phi^{0}_{i}(\psi_{1}) \cap \Phi_{i-1}(\Box^{-} \psi_{1})$     \\
        \Diamond \psi_{1}       & $\Phi^{0}_{i}(\psi_{1}) \cup x^{\Diamond \psi_{1}}_{i}$         \\
        \Diamond^{-} \psi_{1}   & $\Phi^{0}_{i}(\psi_{1}) \cup \Phi_{i-1}(\Diamond^{-} \psi_{1})$ \\ \bottomrule
    \end{tabular*}
    \caption{$\Phi^{0}_{i}(\psi)$ with $\Box, \Box^{-}, \Diamond$ and $\Diamond^{-}$}
    \label{tab:extension/operatorsnotimplemented/phiI}
\end{table}
$\mathsf{Sub}(\phi)$ denotes the set of all TQs occurring as temporal subqueries in $\phi$ (including $\phi$ itself).

\begin{theorem}
    \label{th:extension/operatorsnotimplemented/corretandbounded}
    The extension of the bounded history encoding from \cite{borgwardt2015temporalizing} by $\Box, \Box^{-}, \Diamond$ and $\Diamond^{-}$ preserves correctness and boundedness.
\end{theorem}

\begin{proof}
    To prove that the correctness and boundedness of the algorithm is preserved, the necessary cases are added to the
    corresponding proofs from \cite{borgwardt2015temporalizing}.

    \begin{lemma}[ct.\ Lemma 6.3 in \cite{borgwardt2015temporalizing}]
        \label{lem:extension/operatorsnotimplemented/lemma5}
        The function $\Phi_{0}$ for $\Box, \Box^{-}, \Diamond$ and $\Diamond^{-}$ is correct for 0.
    \end{lemma}

    \begin{proof}
        It is shown by induction on the structure of the subqueries $\psi \in \mathsf{Sub}(\phi)$ that $\mathsf{eval}^{n}(\Phi_{0}(\psi))$
        is equal to $\mathsf{Ans}(\psi, \mathfrak{I}^{(n)},0)$ for all $n \geq 0$. \\
        If $\psi = \Box^{-}\psi_{1}$ or $\psi = \Diamond^{-}\psi_{1}$, then
        \[\(\mathsf{eval}^{n}(\Phi_{0}(\psi)) = \mathsf{eval}^{n}(\Phi_{0}(\psi_{1}))\).\]
        This is by induction equal to $\mathsf{Ans}(\psi_{1}, \mathfrak{I}^{(n)},0)$ which then is, as shown in Proposition REF PROPOSITION XY,
        equal to $\mathsf{Ans}(\psi, \mathfrak{I}^{(n)},0)$.\\
        If $\psi = \Box\psi_{1}$, then
        \begin{equation}
            \notag
            \label{eq:extension/operatorsnotimplemented/proofLemma5/eq1}
            \begin{split}
                \(\mathsf{eval}^{n}(\Phi_{0}(\psi)) &= \mathsf{eval}^{n}(\Phi_{0}(\psi_{1})) \cap \mathsf{eval}^{n}(x^{\psi}_{0}) \\
                &= \mathsf{Ans}(\psi_{1}, \mathfrak{I}^{(n)},0) \cap \left\{\begin{array}{lr}
                                                                                \mathsf{Ans}(\psi, \mathfrak{I}^{(n)},1) & \text{if } n > 0\\
                                                                                \Delta^{\mathsf{N}_\mathsf{V}}       & \text{if } n = 0\\
                \end{array}\right\} \\
                &= \mathsf{Ans}(\psi, \mathfrak{I}^{(n)},0)\)
            \end{split}
        \end{equation}
        If $\psi = \Diamond\psi_{1}$, then
        \begin{equation}
            \notag
            \label{eq:extension/operatorsnotimplemented/proofLemma5/eq2}
            \begin{split}
                \(\mathsf{eval}^{n}(\Phi_{0}(\psi)) &= \mathsf{eval}^{n}(\Phi_{0}(\psi_{1})) \cup \mathsf{eval}^{n}(x^{\psi}_{0}) \\
                &= \mathsf{Ans}(\psi_{1}, \mathfrak{I}^{(n)},0) \cup \left\{\begin{array}{lr}
                                                                                \mathsf{Ans}(\psi, \mathfrak{I}^{(n)},1) & \text{if } n > 0\\
                                                                                \emptyset            & \text{if } n = 0\\
                \end{array}\right\} \\
                &= \mathsf{Ans}(\psi, \mathfrak{I}^{(n)},0)\)
            \end{split}
        \end{equation}
    \end{proof}

    \begin{lemma}[ct.\ Lemma 6.4 in \cite{borgwardt2015temporalizing}]
        \label{lem:extension/operatorsnotimplemented/lemma6}
        If $\Phi_{i-1}$ for $\Box, \Box^{-}, \Diamond$ and $\Diamond^{-}$ is correct for i-1, then $\Phi^{0}_{i}$ for $\Box, \Box^{-}, \Diamond$ and $\Diamond^{-}$ is correct for i.
    \end{lemma}

    \begin{proof}
        It is shown by induction on the structure of the subqueries $\psi \in \mathsf{Sub}(\phi)$ that $\mathsf{eval}^{n}(\Phi^{0}_{i}(\psi))$
        is equal to $\mathsf{Ans}(\psi, \mathfrak{I}^{(n)},i)$ for all $n \geq i$. \\
        If $\psi = \Box^{-}\psi_{1}$, then
        \begin{equation}
            \notag
            \label{eq:extension/operatorsnotimplemented/proofLemma6/eq1}
            \begin{split}
                \(\mathsf{eval}^{n}(\Phi^{0}_{i}(\psi)) &= \mathsf{eval}^{n}(\Phi^{0}_{i}(\psi_{1})) \cap \mathsf{eval}^{n}(\Phi_{i-1}(\psi)) \\
                &= \mathsf{Ans}(\psi_{1}, \mathfrak{I}^{(n)},i) \cap \mathsf{Ans}(\psi, \mathfrak{I}^{(n)},i-1) \\
                &= \mathsf{Ans}(\psi, \mathfrak{I}^{(n)},i)\)
            \end{split}
        \end{equation}
        If $\psi = \Diamond^{-}\psi_{1}$, then
        \begin{equation}
            \notag
            \label{eq:extension/operatorsnotimplemented/proofLemma6/eq2}
            \begin{split}
                \(\mathsf{eval}^{n}(\Phi^{0}_{i}(\psi)) &= \mathsf{eval}^{n}(\Phi^{0}_{i}(\psi_{1})) \cup \mathsf{eval}^{n}(\Phi_{i-1}(\psi)) \\
                &= \mathsf{Ans}(\psi_{1}, \mathfrak{I}^{(n)},i) \cup \mathsf{Ans}(\psi, \mathfrak{I}^{(n)},i-1) \\
                &= \mathsf{Ans}(\psi, \mathfrak{I}^{(n)},i)\)
            \end{split}
        \end{equation}
        If $\psi = \Box\psi_{1}$, then
        \begin{equation}
            \notag
            \label{eq:extension/operatorsnotimplemented/proofLemma6/eq3}
            \begin{split}
                \(\mathsf{eval}^{n}(\Phi^{0}_{i}(\psi)) &= \mathsf{eval}^{n}(\Phi^{0}_{i}(\psi_{1})) \cap \mathsf{eval}^{n}(x^{\psi}_{i}) \\
                &= \mathsf{Ans}(\psi_{1}, \mathfrak{I}^{(n)},i) \cap \left\{\begin{array}{lr}
                                                                                \mathsf{Ans}(\psi, \mathfrak{I}^{(n)},i+1)  & \text{if } n > i\\
                                                                                \Delta^{\mathsf{N}_\mathsf{V}}          & \text{if } n = i\\
                \end{array}\right\} \\
                &= \mathsf{Ans}(\psi, \mathfrak{I}^{(n)},i)\)
            \end{split}
        \end{equation}
        If $\psi = \Diamond\psi_{1}$, then
        \begin{equation}
            \notag
            \label{eq:extension/operatorsnotimplemented/proofLemma6/eq4}
            \begin{split}
                \(\mathsf{eval}^{n}(\Phi^{0}_{i}(\psi)) &= \mathsf{eval}^{n}(\Phi^{0}_{i}(\psi_{1})) \cup \mathsf{eval}^{n}(x^{\psi}_{i}) \\
                &= \mathsf{Ans}(\psi_{1}, \mathfrak{I}^{(n)},i) \cup \left\{\begin{array}{lr}
                                                                                \mathsf{Ans}(\psi, \mathfrak{I}^{(n)},i+1)  & \text{if } n > i\\
                                                                                \emptyset               & \text{if } n = i\\
                \end{array}\right\} \\
                &= \mathsf{Ans}(\psi, \mathfrak{I}^{(n)},i)\)
            \end{split}
        \end{equation}
    \end{proof}

    \begin{lemma}[ct.\ Lemma 6.5 in \cite{borgwardt2015temporalizing}]
        \label{lem:extension/operatorsnotimplemented/lemma7}
        If $\Phi_{i-1}$ for $\Box, \Box^{-}, \Diamond$ and $\Diamond^{-}$ is correct for i-1 and (i-1)-bounded, then we can construct a function $\Phi_{i}: \mathsf{Sub}(\phi)
        \rightarrow \mathsf{AT}^{i}_{\phi}$ for $\Box, \Box^{-}, \Diamond$ and $\Diamond^{-}$ that is correct for i and i-bounded.
    \end{lemma}

    \begin{proof}
        The in \cite{borgwardt2015temporalizing} introduced function $\mathsf{update}(x^{\psi^{j}}_{i-1})$ needs to be extended, before it then can be shown for all $n \geq i$ that
        $\mathsf{eval}^{n}(x^{\psi^{j}}_{i-1})$ is still equal to $\mathsf{eval}^{n}(\mathsf{update}(x^{\psi^{j}}_{i-1}))$.
        After considering the new operators, $\mathsf{update}(x^{\psi^{j}}_{i-1})$ looks like this:
        \begin{equation}
            \notag
            \label{eq:extension/operatorsnotimplemented/proofLemma7/eq1}
            \begin{split}
                \(\mathsf{update}(x^{\psi^{j}}_{i-1}) := \left\{\begin{array}{lr}
                                                                    \Phi^{j-1}_{i}(\psi_{1})  & \text{if } \psi^{j}=\Circle\psi_{1} \text{ or } \psi^{j}=\CIRCLE\psi_{1}\\
                                                                    \Phi^{j-1}_{i}(\psi^{j})  & \text{if } \psi^{j}=\psi_{1}\mathsf{U}\psi_{2} \text{ or } \psi^{j}=\Box\psi_{1} \text{ or } \psi^{j}=\Diamond\psi_{1}\\
                \end{array}\right\}\)
            \end{split}
        \end{equation}
        For $\psi^{j}=\Box\psi_{1}$ and $\psi^{j}=\Diamond\psi_{1}$, by definition $\mathsf{eval}^{n}(x^{\psi^{j}}_{i-1}) = \mathsf{Ans}(\psi^{j}, \mathfrak{I}^{(n)},i)$.
        Since $\Phi^{j-1}_{i}$ is correct for $i$, this is the same set as $\mathsf{eval}^{n}(\Phi^{j-1}_{i}(\psi^{j}))=\mathsf{eval}^{n}(\mathsf{update}(x^{\psi^{j}}_{i-1}))$.

        It remains to show $i$-boundedness of $\Phi_{i}=\Phi^{k}_{i}$.
        In \cite{borgwardt2015temporalizing} this is again proven by induction on $j$.
        It therefore suffices to add the missing cases.
        It is enough to show that $\mathsf{update}(x^{\psi^{j}}_{i-1})$ contains only variables from $\mathsf{Var}^{\psi^{j}}_{i}$.
        If $\psi^{j}=\Box\psi_{1}$ or $\psi^{j}=\Diamond\psi_{1}$, then $\mathsf{update}(x^{\psi^{j}}_{i-1}) = \Phi^{j-1}_{i}(\psi^{j})$.
        Since $\Phi^{j-1}_{i}$ differs from $\Phi^{0}_{i}$ only in the replacement of some variables with index $i-1$,
        \begin{align}
            &\Phi^{j-1}_{i}(\psi^{j}) = \Phi^{j-1}_{i}(\psi_{1}) \cap x^{\psi^{j}}_{i} \nonumber \\
            \text{or} \nonumber \\
            &\Phi^{j-1}_{i}(\psi^{j}) = \Phi^{j-1}_{i}(\psi_{1}) \cup x^{\psi^{j}}_{i} \text{, respectively.} \nonumber
        \end{align}

        By the induction hypothesis $\Phi^{j-1}_{i}(\psi_{1})$ contains only variables from $\mathsf{Var}^{\psi_{1}}_{i} = \mathsf{Var}^{\psi^{j}}_{i} \setminus \{x^{\psi^{j}}_{i}\}$
        and $\mathsf{Var}^{\psi_{1}}_{i-1} \cap \{x^{\psi^{j}}_{i-1},\dots,x^{\psi^{k}}_{i-1}\}$.
        Since every variable $x^{\psi^{'}}_{i-1} \in \mathsf{Var}^{\psi_{1}}_{i-1}$ must satisfy $\psi^{'} \in \mathsf{FSub}(\psi_{1})$ the second set $\mathsf{Var}^{\psi_{1}}_{i-1} \cap \{x^{\psi^{j}}_{i-1},...,x^{\psi^{k}}_{i-1}\}$
        is empty.
        This follows from the total order $\psi^{1} \prec \dots \prec \psi^{k}$ on the set $\mathsf{FSub}(\phi) = \{\psi^{1} , \dots , \psi^{k}\}$ presented in \cite{borgwardt2015temporalizing},
        i.e. $\psi^{'} \in \mathsf{FSub}(\psi^{j}) \setminus \{\psi^{j}\}$, and thus $\psi^{'} \prec \psi^{j}$.
    \end{proof}

    This concludes the proof of Theorem REF THEOREM XY.
\end{proof}


\subsection{Filtering of query results}
\label{subsec:extension/extensions/filtering}
Second, the algorithm presented in \cite{borgwardt2015temporalizing} is extended by a filter operator.
This filter operator $f[\phi]$ denotes that the filter query $f$ is applied to the result of $\phi$.
Since $\mathcal{Q}$-queries are not further specified, it is impossible to determine at which position in $f$ the result of $\phi$ should be inserted.
However, it is known that the implementation of the BHE \cite{boundedhistoryencodingalgorithm} is based on SQL.
Therefore it can be defined that $f$ is an SQL statement and the result of $\phi$ is inserted at ``SELECT * FROM $\phi$''.

The semantics of TQs from Definition \ref{def:Extension/operatorsnotimplemented/semantics} are extended as follows:
\begin{definition}[semantics of temporal queries with a filter operator ct.\ Definition 3.3 in \cite{borgwardt2015temporalizing}]
    \label{def:extension/filtering/semantics}
    Let $\phi$ be a TQ, $\mathfrak{I} = (I_{i})_{0 \leq i \leq n}$ a sequence of interpretations over a common domain,
    $\mathfrak{a}:\mathsf{FVar}(\phi) \rightarrow \mathsf{N}_{\mathsf{C}}$ a variable assignment, $f$ be an SQL statement, and $i$ be an integer with $0 \leq i \leq n$.
    The \textit{satisfaction relation} $\mathfrak{I}, i \models \mathfrak{a}(\phi)$ is defined by induction on the structure of $\phi$ as follows:
    \begin{table}[H]
        \centering
        \begin{tabular*}{\textwidth}{@{}ll@{}}
            \toprule
            $\phi$                            & $\mathfrak{I}, i \models \mathfrak{a}(\phi)$ iff  \\ \midrule
            $[\ldots]$                & $[\ldots]$ \\
            $f[\phi]$                       & $I_{i} \models \mathfrak{a}(f)$ and $\mathfrak{I},i \models \mathfrak{a}(\phi)$            \\ \bottomrule
        \end{tabular*}
        \caption{Satisfaction relation of temporal queries with a filter operator}
        \label{tab:extension/filtering/satisfactionrelation}
    \end{table}
\end{definition}
Now Query \ref{qu:queries/investigationqueries/practical/qu2} can be rewritten as follows:
\begin{align}
    &\text{``SELECT * FROM $(\Circle^{-}$(SELECT $marke$, AVG($price$) AS $price1$ FROM $autos$) $\wedge$}  \nonumber \\
    &\text{SELECT $marke$, AVG($price$) AS $price2$ FROM $autos$) WHERE $price1 < price2$''}.  \nonumber
\end{align}

The semantics of the filter operator can now be used to extend the algorithm specified in \cite{borgwardt2015temporalizing}.
The functions $\mathsf{eval}^{n}: \mathsf{AT}^{n}_{\phi} \rightarrow 2^{\Delta^{\mathsf{N}_\mathsf{V}}}, n \geq 0$ in \cite{borgwardt2015temporalizing} do not have to be extended.

The function $\Phi_{0}(\psi): \mathsf{Sub}(\phi) \rightarrow \mathsf{AT}^{0}_{\phi}$ in \cite{borgwardt2015temporalizing} has to be extended as follows:
\begin{table}[H]
    \centering
    \begin{tabular*}{\textwidth}{@{}ll@{}}
        \toprule
        $\psi$            & $\Phi_{0}(\psi)$                                      \\ \midrule
        $[\ldots]$                & $[\ldots]$ \\
        $f[\psi_{1}]$      & $\mathsf{Ans}(f[\Phi_{0}(\psi_{1})], I_{0})$     \\\bottomrule
    \end{tabular*}
    \caption{$\Phi_{0}(\psi)$ with a filter operator}
    \label{tab:extension/filtering/phi0}
\end{table}

The function $\Phi^{0}_{i}(\psi): \mathsf{Sub}(\phi) \rightarrow \mathsf{AT}^{i}_{\phi},\ i>0$ in \cite{borgwardt2015temporalizing} has to be extended as follows:
\begin{table}[H]
    \centering
    \begin{tabular*}{\textwidth}{@{}ll@{}}
        \toprule
        $\psi$            & $\Phi_{0}(\psi)$                                      \\ \midrule
        $[\ldots]$                & $[\ldots]$ \\
        $f[\psi_{1}]$      & $\mathsf{Ans}(f[\Phi^{0}_{i}(\psi_{1})], I_{i})$     \\\bottomrule
    \end{tabular*}
    \caption{$\Phi^{0}_{i}(\psi)$ with a filter operator}
    \label{tab:extension/filtering/phiI}
\end{table}

\begin{theorem}
    \label{th:extension/filtering/corretandbounded}
    Extending the algorithm from \cite{borgwardt2015temporalizing} by a filter operator preserves correctness and boundedness.
\end{theorem}

\begin{proof}
    To prove that the correctness and boundedness of the algorithm is preserved, the necessary cases are added to the
    corresponding proofs from \cite{borgwardt2015temporalizing}.

    \begin{lemma}[ct.\ Lemma 6.3 in \cite{borgwardt2015temporalizing}]
        \label{lem:extension/filtering/lemma5}
        The function $\Phi_{0}$ with a filter operator is correct for 0.
    \end{lemma}

    \begin{proof}
        It is shown by induction on the structure of the subqueries $\psi \in \mathsf{Sub}(\phi)$ that $\mathsf{eval}^{n}(\Phi_{0}(\psi))$
        is equal to $\mathsf{Ans}(\psi, \mathfrak{I}^{(n)},0)$ for all $n \geq 0$. \\
        If $\psi = f[\psi_{1}]$, then
        \[\mathsf{eval}^{n}(\Phi_{0}(\psi)) = \mathsf{Ans}(f[\Phi_{0}(\psi_{1})], I_{0}) = \mathsf{Ans}(\psi, \mathfrak{I}^{(n)},0).\]
    \end{proof}

    \begin{lemma}[ct.\ Lemma 6.4 in \cite{borgwardt2015temporalizing}]
        \label{lem:extension/filtering/lemma6}
        If $\Phi_{i-1}$ with a filter operator is correct for i-1, then $\Phi^{0}_{i}$ with a filter operator is correct for i.
    \end{lemma}

    \begin{proof}
        It is shown by induction on the structure of the subqueries $\psi \in \mathsf{Sub}(\phi)$ that $\mathsf{eval}^{n}(\Phi^{0}_{i}(\psi))$
        is equal to $\mathsf{Ans}(\psi, \mathfrak{I}^{(n)},i)$ for all $n \geq i$. \\
        If $\psi = f[\psi_{1}]$, then
        \[\mathsf{eval}^{n}(\Phi^{0}_{i}(\psi)) = \mathsf{Ans}(f[\Phi^{0}_{i}(\psi_{1})], I_{i}) = \mathsf{Ans}(\psi, \mathfrak{I}^{(n)},i).\]
    \end{proof}

    \begin{lemma}[ct.\ Lemma 6.5 in \cite{borgwardt2015temporalizing}]
        \label{lem:extension/filtering/lemma7}
        If $\Phi_{i-1}$ with a filter operator is correct for i-1 and (i-1)-bounded, then we can construct a function $\Phi_{i}: \mathsf{Sub}(\phi)
        \rightarrow \mathsf{AT}^{i}_{\phi}$ with a filter operator that is correct for i and i-bounded.
    \end{lemma}

    \begin{proof}
        Since $f[\psi_{1}]$ introduces no new variables, this follows directly from Lemma 6.5 in \cite{borgwardt2015temporalizing}.
    \end{proof}

    This concludes the proof of Theorem \ref{th:extension/filtering/corretandbounded}.
\end{proof}


\subsection{Metric Temporal Operators}
\label{subsec:extension/extensions/metrictemporaloperators}
Third, the algorithm presented in \cite{borgwardt2015temporalizing} is extended by MTOs.
For $\Circle \phi$, $\CIRCLE \phi$, $\Circle^{-} \phi$ and $\CIRCLE^{-} \phi$ the MTOs simplify composition, e.g. $\Circle_{3} \phi = \Circle (\Circle (\Circle \phi))$.
For $\Box \phi$, $\Box^{-} \phi$, $\Diamond \phi$, $\Diamond^{-} \phi$, $\phi_{1} \mathsf{U} \phi_{2}$, and $\phi_{1} \mathsf{S} \phi_{2}$ MTOs restrict the number of time points,
e.g. $\Box_{5} \phi$ (for 5 time points) or $\Diamond_{5} \phi$ (some time in 5 time points).

The semantics of TQs from Definition \ref{def:Extension/operatorsnotimplemented/semantics} are extended as follows:
\begin{definition}[semantics of temporal queries with metric temporal operators ct.\ Definition 3.3 in \cite{borgwardt2015temporalizing}]
    \label{def:extension/metrictemporaloperators/semantics}
    Let $\phi$ be a TQ, $\mathfrak{I} = (I_{i})_{0 \leq i \leq n}$ a sequence of interpretations over a common domain,
    $\mathfrak{a}:\mathsf{FVar}(\phi) \rightarrow \mathsf{N}_{\mathsf{C}}$ a variable assignment, $i$ be an integer with $0 \leq i \leq n$, and $p$ be an integer with $p \geq 0$.
    \begin{minipage}[H]{\textwidth}
        The \textit{satisfaction relation} $\mathfrak{I}, i \models \mathfrak{a}(\phi)$ is defined by induction on the structure of $\phi$ as follows:
        \begin{table}[H]
            \centering
            \begin{tabular*}{\textwidth}{@{}ll@{}}
                \toprule
                $\phi$                            & $\mathfrak{I}, i \models \mathfrak{a}(\phi)$ iff                                                 \\ \midrule
                $[\ldots]$                      & $[\ldots]$ \\
                $\Circle_{p} \phi_{1}$            & $i+p \leq n$ and $\mathfrak{I},i+p \models \mathfrak{a}(\phi_{1})$                              \\
                $\CIRCLE_{p} \phi_{1}$            & $i+p \leq n$ implies $\mathfrak{I},i+p \models \mathfrak{a}(\phi_{1})$                          \\
                $\Circle^{-}_{p} \phi_{1}$        & $i-p \geq 0$ and $\mathfrak{I},i-p \models \mathfrak{a}(\phi_{1})$                              \\
                $\CIRCLE^{-}_{p} \phi_{1}$        & $i-p \geq 0$ implies $\mathfrak{I},i-p \models \mathfrak{a}(\phi_{1})$                          \\
                $\Box_{p} \phi_{1}$               & $\mathfrak{I}, k \models \mathfrak{a}(\phi_{1})$ for all $k$, $i \leq k \leq \mathsf{min}(i+p,n)$                  \\
                $\Box^{-}_{p} \phi_{1}$           & $\mathfrak{I}, k \models \mathfrak{a}(\phi_{1})$ for all $k$, $\mathsf{max}(i-p,0) \leq k \leq i$                  \\
                $\Diamond_{p} \phi_{1}$           & $\mathfrak{I}, k \models \mathfrak{a}(\phi_{1})$ for some $k$, $i \leq k \leq \mathsf{min}(i+p,n)$                 \\
                $\Diamond^{-}_{p} \phi_{1}$       & $\mathfrak{I}, k \models \mathfrak{a}(\phi_{1})$ for some $k$, $\mathsf{max}(i-p,0) \leq k \leq i$                 \\
                $\phi_{1} \mathsf{U}_{p} \phi_{2}$& there is $k$, $i \leq k \leq \mathsf{min}(i+p,n)$, with $\mathfrak{I}, k \models \mathfrak{a}_{\phi_{2}}(\phi_{2})$   \\
                & and $\mathfrak{I}, j \models \mathfrak{a}_{\phi_{1}}(\phi_{1})$ for all $j, i \leq j < k$          \\
                $\phi_{1} \mathsf{S}_{p} \phi_{2}$& there is $k$, $\mathsf{max}(i-p,0) \leq k \leq i$, with $\mathfrak{I}, k \models \mathfrak{a}_{\phi_{2}}(\phi_{2})$   \\
                & and $\mathfrak{I}, j \models \mathfrak{a}_{\phi_{1}}(\phi_{1})$ for all $j, k < j \leq i$          \\ \bottomrule
            \end{tabular*}
            \caption{Satisfaction relation of temporal queries with metric temporal operators}
            \label{tab:extension/metrictemporaloperators/satisfactionrelation}
        \end{table}
    \end{minipage}
    \\

    $\mathfrak{a}_{\phi_{1}}$ denotes the restriction of a variable assignment $\mathfrak{a}:\mathsf{FVar}(\phi) \rightarrow \mathsf{N}_{\mathsf{C}}$ to $\mathsf{FVar}(\phi_{1})$
    for a subquery $\phi_{1}$ of $\phi$.
\end{definition}
Now Query \ref{qu:queries/investigationqueries/practical/qu3} can be rewritten as follows:
\begin{align}
    &\text{``$\Diamond^{-}_{6}$(SELECT $url$ FROM $autos$ WHERE $price$ $>$ 1000000)''} \nonumber
\end{align}

\begin{restatable}[ct.\ Proposition 3.4 in \cite{borgwardt2015temporalizing}]{proposition}{propMTOsPzero}
    \label{prop:extension/metrictemporaloperators/prop2}
    For $\mathfrak{a}:\mathsf{FVar}(\phi) \rightarrow \mathsf{N}_{\mathsf{C}}$, $0 \leq i \leq n$ and $p = 0$,
    $\mathfrak{I}, i \models \mathfrak{a}(\phi)$ iff $\mathfrak{I}, i \models \mathfrak{a}(\phi_{1})$ or $\mathfrak{I}, i \models \mathfrak{a}_{\phi_{2}}(\phi_{2})$, respectively.
\end{restatable}

\begin{proof}
    To prove the above proposition, three equivalences are demonstrated here.
    The missing cases work similarly and can be found in Appendix \ref{ch:appendixC}.
    The proof works mainly based on semantics.
    \begin{itemize}
        \item $\Circle_{0} \phi_{1} \equiv \phi_{1}$
        \begin{align}
            &\mathfrak{I},i \models \mathfrak{a}(\Circle_{0} \phi_{1}) \\
            \Leftrightarrow &i+0 \leq n\text{ and }\mathfrak{I},i+0 \models \mathfrak{a}(\phi_{1}) \\
            \Leftrightarrow &\mathfrak{I}, i \models \mathfrak{a}(\phi_{1})
        \end{align}

        \item $\Box^{-}_{0} \phi_{1} \equiv \phi_{1}$
        \begin{align}
            &\mathfrak{I}, i \models \mathfrak{a}(\Box^{-}_{0} \phi_{1}) \\
            \Leftrightarrow &\mathfrak{I}, k \models \mathfrak{a}(\phi_{1}) \text{ for all } k, \mathsf{max}(i-0,0) \leq k \leq i \\
            \Leftrightarrow &\mathfrak{I}, i \models \mathfrak{a}(\phi_{1})
        \end{align}

        \item $\phi_{1} \mathsf{U}_{0} \phi_{2} \equiv \phi_{2}$
        \begin{align}
            &\mathfrak{I}, i \models \mathfrak{a}(\phi_{1} \mathsf{U}_{0} \phi_{2}) \\
            \Leftrightarrow &\text{there is } k, i \leq k \leq \mathsf{min}(i+0,n)\text{, with } \mathfrak{I}, k \models \mathfrak{a}_{\phi_{2}}(\phi_{2}) \text{ and } \\
            &\mathfrak{I}, j \models \mathfrak{a}_{\phi_{1}}(\phi_{1})\text{ for all }j, i \leq j < k \nonumber \\
            \Leftrightarrow &\mathfrak{I}, i \models \mathfrak{a}_{\phi_{2}}(\phi_{2})
        \end{align}
    \end{itemize}
\end{proof}

There are again equivalences similar to Proposition \ref{prop:extension/operatorsnotimplemented/prop2}.
\begin{itemize}
    \item $\Circle_{p} \phi_{1}$ is equivalent to $\Circle \Circle_{p-1} \phi_{1}$;
    \item $\CIRCLE_{p} \phi_{1}$ is equivalent to $\CIRCLE \CIRCLE_{p-1} \phi_{1}$;
    \item $\Circle^{-}_{p} \phi_{1}$ is equivalent to $\Circle^{-} \Circle^{-}_{p-1} \phi_{1}$;
    \item $\CIRCLE^{-}_{p} \phi_{1}$ is equivalent to $\CIRCLE^{-} \CIRCLE^{-}_{p-1} \phi_{1}$;
    \item $\Box_{p} \phi_{1} $ is equivalent to $ \phi_{1} \wedge \CIRCLE \Box_{p-1} \phi_{1}$;
    \item $\Box^{-}_{p} \phi_{1} $ is equivalent to $ \phi_{1} \wedge \CIRCLE^{-} \Box^{-}_{p} \phi_{1}$;
    \item $\Diamond_{p} \phi_{1} $ is equivalent to $ \phi_{1} \vee \Circle \Diamond_{p} \phi_{1}$;
    \item $\Diamond^{-}_{p} \phi_{1} $ is equivalent to $ \phi_{1} \vee \Circle^{-} \Diamond^{-}_{p} \phi_{1}$;
    \item $\phi_{1} \mathsf{U}_{p} \phi_{2}$ is equivalent to $\phi_{2} \vee (\phi_{1} \wedge \Circle (\phi_{1} \mathsf{U}_{p-1} \phi_{2}))$;
    \item $\phi_{1} \mathsf{S}_{p} \phi_{2}$ is equivalent to $\phi_{2} \vee (\phi_{1} \wedge \Circle^{-} (\phi_{1} \mathsf{S}_{p-1} \phi_{2}))$.
\end{itemize}
Thus, at the last time point
\begin{itemize}
    \item $\Circle_{p} \phi_{1}$ does not have any answers because $\Circle \Circle_{p-1} \phi_{1}$ does not have any answers,
    \item $\CIRCLE_{p} \phi_{1}$ is tautological because $\CIRCLE \CIRCLE_{p-1} \phi_{1}$ is tautological,
    \item $\Box_{p} \phi_{1} $ is equivalent to $ \phi_{1}$ because $\CIRCLE \Box_{p-1} \phi_{1}$ is tautological,
    \item $\Diamond_{p} \phi_{1} $ is equivalent to $ \phi_{1}$ because $\Circle \Diamond_{p} \phi_{1}$ does not have any answers and
    \item $\phi_{1} \mathsf{U}_{p} \phi_{2}$ is equivalent to $\phi_{2}$ because $\Circle (\phi_{1} \mathsf{U}_{p-1} \phi_{2})$ does not have any answers,
\end{itemize}
and at the first time point
\begin{itemize}
    \item $\Circle^{-}_{p} \phi_{1}$ does not have any answers because $\Circle^{-} \Circle^{-}_{p-1} \phi_{1}$ does not have any answers,
    \item $\CIRCLE^{-}_{p} \phi_{1}$ is tautological because $\CIRCLE^{-} \CIRCLE^{-}_{p-1} \phi_{1}$ is tautological,
    \item $\Box^{-}_{p} \phi_{1} $ is equivalent to $ \phi_{1}$ because $\CIRCLE^{-} \Box^{-}_{p} \phi_{1}$ is tautological,
    \item $\Diamond^{-}_{p} \phi_{1} $ is equivalent to $ \phi_{1}$ because $\Circle^{-} \Diamond^{-}_{p} \phi_{1}$ does not have any answers and
    \item $\phi_{1} \mathsf{S}_{p} \phi_{2}$ is equivalent to $\phi_{2}$ because $\Circle^{-} (\phi_{1} \mathsf{S}_{p-1} \phi_{2})$ does not have any answers.
\end{itemize}

\begin{restatable}[ct.\ Proposition 3.4 in \cite{borgwardt2015temporalizing}]{proposition}{propMTOs}
    \label{prop:extension/metrictemporaloperators/prop3}
    For $\mathfrak{a}:\mathsf{FVar}(\phi) \rightarrow \mathsf{N}_{\mathsf{C}}$, $0 \leq i \leq n$ and $p > 0$,
    \begin{enumerate}
        \item $\mathfrak{I}, i \models \mathfrak{a}(\emph{\Circle}_{p} \phi_{1})$ iff
        \begin{itemize}
            \item $i < n$ and $\mathfrak{I}, i+1 \models \mathfrak{a}(\emph{\Circle}_{p-1} \phi_{1})$
        \end{itemize}
        \item $\mathfrak{I}, i \models \mathfrak{a}(\emph{\CIRCLE}_{p} \phi_{1})$ iff
        \begin{itemize}
            \item $i < n$ implies $\mathfrak{I}, i+1 \models \mathfrak{a}(\emph{\CIRCLE}_{p-1} \phi_{1})$
        \end{itemize}
        \item $\mathfrak{I}, i \models \mathfrak{a}(\emph{\Circle}^{-}_{p} \phi_{1})$ iff
        \begin{itemize}
            \item $i > 0$ and $\mathfrak{I}, i-1 \models \mathfrak{a}(\emph{\Circle}^{-}_{p-1} \phi_{1})$
        \end{itemize}
        \item $\mathfrak{I}, i \models \mathfrak{a}(\emph{\CIRCLE}^{-}_{p} \phi_{1})$ iff
        \begin{itemize}
            \item $i > 0$ implies $\mathfrak{I}, i-1 \models \mathfrak{a}(\emph{\CIRCLE}^{-}_{p-1} \phi_{1})$
        \end{itemize}
        \item $\mathfrak{I}, i \models \mathfrak{a}(\Box_{p} \phi_{1})$ iff
        \begin{itemize}
            \item $\mathfrak{I}, i \models \mathfrak{a}(\phi_{1})$ and
            \item $i < n$ implies $\mathfrak{I}, i+1 \models \mathfrak{a}(\Box_{p-1} \phi_{1})$
        \end{itemize}
        \item $\mathfrak{I}, i \models \mathfrak{a}(\Box^{-}_{p} \phi_{1})$ iff
        \begin{itemize}
            \item $\mathfrak{I}, i \models \mathfrak{a}(\phi_{1})$ and
            \item $i > 0$ implies $\mathfrak{I}, i-1 \models \mathfrak{a}(\Box^{-}_{p-1} \phi_{1})$
        \end{itemize}
        \item $\mathfrak{I}, i \models \mathfrak{a}(\Diamond_{p} \phi_{1})$ iff
        \begin{itemize}
            \item $\mathfrak{I}, i \models \mathfrak{a}(\phi_{1})$ or
            \item $i < n$ and $\mathfrak{I}, i+1 \models \mathfrak{a}(\Diamond_{p-1} \phi_{1})$
        \end{itemize}
        \item $\mathfrak{I}, i \models \mathfrak{a}(\Diamond^{-}_{p} \phi_{1})$ iff
        \begin{itemize}
            \item $\mathfrak{I}, i \models \mathfrak{a}(\phi_{1})$ or
            \item $i > 0$ and $\mathfrak{I}, i-1 \models \mathfrak{a}(\Diamond^{-}_{p-1} \phi_{1})$
        \end{itemize}
        \item $\mathfrak{I}, i \models \mathfrak{a}(\phi_{1} \mathsf{U}_{p} \phi_{2})$ iff
        \begin{itemize}
            \item $\mathfrak{I}, i \models \mathfrak{a}_{\phi_{2}}(\phi_{2})$ or
            \item $\mathfrak{I}, i \models \mathfrak{a}_{\phi_{1}}(\phi_{1})$ and $i < n$ and $\mathfrak{I}, i+1 \models \mathfrak{a}(\phi_{1} \mathsf{U}_{p-1} \phi_{2})$
        \end{itemize}
        \item $\mathfrak{I}, i \models \mathfrak{a}(\phi_{1} \mathsf{S}_{p} \phi_{2})$ iff
        \begin{itemize}
            \item $\mathfrak{I}, k \models \mathfrak{a}_{\phi_{2}}(\phi_{2})$ or
            \item $\mathfrak{I}, i \models \mathfrak{a}_{\phi_{1}}(\phi_{1})$ and $i > 0$ and $\mathfrak{I}, i-1 \models \mathfrak{a}(\phi_{1} \mathsf{S}_{p-1} \phi_{2})$
        \end{itemize}
    \end{enumerate}
\end{restatable}

\begin{proof}
    To prove the above proposition, three equivalences are demonstrated here.
    The missing cases work similarly and can be found in Appendix \ref{ch:appendixC}.
    The proof works mainly based on semantics.
    \begin{itemize}
        \item[1.] $\Circle_{p} \phi_{1} \equiv \Circle \Circle_{p-1} \phi_{1}$
        \begin{align}
            &\mathfrak{I}, i \models \mathfrak{a}(\Circle_{p} \phi_{1}) \\
            \Leftrightarrow &i + p \leq n \text{ and }\mathfrak{I}, i+p \models \mathfrak{a}(\phi_{1}) \label{eq:extension/metrictemporaloperators/proofProp3/eq1.2}\\
            \Leftrightarrow & i < n\text{ and }(i+1)+(p-1) \leq n \text{ and }\mathfrak{I}, (i+1)+(p-1) \models \mathfrak{a}(\phi_{1}) \label{eq:extension/metrictemporaloperators/proofProp3/eq1.3}\\
            \Leftrightarrow & i < n\text{ and }\mathfrak{I}, i+1 \models \mathfrak{a}(\Circle_{p-1} \phi_{1}) \\
            \Leftrightarrow &\mathfrak{I}, i \models \mathfrak{a}(\Circle \Circle_{p-1} \phi_{1})
        \end{align}

        \item[6.] $\Box^{-}_{p} \phi_{1} \equiv \phi_{1} \wedge \CIRCLE^{-} \Box^{-}_{p} \phi_{1}$
        \begin{align}
            &\mathfrak{I}, i \models \mathfrak{a}(\Box^{-}_{p} \phi_{1}) \\
            \Leftrightarrow &\mathfrak{I}, k \models \mathfrak{a}(\phi_{1}) \text{ for all } k, \mathsf{max}(i-p,0) \leq k \leq i \label{eq:extension/metrictemporaloperators/proofProp3/eq6.2}\\
            \Leftrightarrow &\mathfrak{I}, i \models \mathfrak{a}(\phi_{1}) \text{ and } (i > 0 \text{ implies } \label{eq:extension/metrictemporaloperators/proofProp3/eq6.3}\\
            &\mathfrak{I}, k \models \mathfrak{a}(\phi_{1}) \text{ for all } k, \mathsf{max}((i-1)-(p-1),0) \leq k \leq i-1) \nonumber \\
            \Leftrightarrow &\mathfrak{I}, i \models \mathfrak{a}(\phi_{1}) \text{ and } (i > 0 \text{ implies } \mathfrak{I}, i-1 \models \mathfrak{a}(\Box^{-}_{p-1} \phi_{1})) \\
            \Leftrightarrow &\mathfrak{I}, i \models \mathfrak{a}(\phi_{1} \wedge \CIRCLE^{-} \Box^{-}_{p-1} \phi_{1})
        \end{align}
        \eqref{eq:extension/metrictemporaloperators/proofProp3/eq6.3} is equivalent to \eqref{eq:extension/metrictemporaloperators/proofProp3/eq6.2} because
        \begin{itemize}
            \item in case $i > 0$, the query needs to be satisfied now, at time point $i$, and at past time points $k$, $\mathsf{max}((i-1)-(p-1),0) \leq k \leq i-1$, to be satisfied.
            Since $i > 0$ is true, the satisfaction of past time points depends solely on the second part of the ``implies''-statement; and
            \item in case $i = 0$, the query needs to be satisfied now, at time point $i$, to be satisfied.
            There are no past time points $k$, $\mathsf{max}((i-1)-(p-1),0) \leq k \leq 0-1$.
            Since $i = 0$ is true, $\mathfrak{I}, i \models \mathfrak{a}(\phi_{1})$ is equivalent to $\mathfrak{I}, k \models \mathfrak{a}(\phi_{1})$ for all $k$, $\mathsf{max}(i-p,0) \leq k \leq 0$,
            and $i > 0$ is not true, thus the ``implies''-statement does not affect satisfaction.
        \end{itemize}

        \item[9.] $\phi_{1} \mathsf{U}_{p} \phi_{2} \equiv \phi_{2} \vee (\phi_{1} \wedge \Circle (\phi_{1} \mathsf{U}_{p-1} \phi_{2}))$
        \begin{align}
            &\mathfrak{I}, i \models \mathfrak{a}(\phi_{1} \mathsf{U}_{p} \phi_{2}) \\
            \Leftrightarrow &\text{there is } k, i \leq k \leq \mathsf{min}(i+p,n)\text{, with } \mathfrak{I}, k \models \mathfrak{a}_{\phi_{2}}(\phi_{2}) \text{ and } \label{eq:extension/metrictemporaloperators/proofProp3/eq9.2}\\
            &\mathfrak{I}, j \models \mathfrak{a}_{\phi_{1}}(\phi_{1})\text{ for all }j, i \leq j < k \nonumber \\
            \Leftrightarrow &\mathfrak{I}, i \models \mathfrak{a}_{\phi_{2}}(\phi_{2})\text{ or }(\mathfrak{I}, i \models \mathfrak{a}_{\phi_{1}}(\phi_{1})\text{ and }(i < n\text{ and } \label{eq:extension/metrictemporaloperators/proofProp3/eq9.3}\\
            &\text{there is } k, i+1 \leq k \leq \mathsf{min}((i+1)+(p-1),n)\text{, with }  \nonumber \\
            &\mathfrak{I}, k \models \mathfrak{a}_{\phi_{2}}(\phi_{2}) \text{ and }\mathfrak{I}, j \models \mathfrak{a}_{\phi_{1}}(\phi_{1})\text{ for all }j, i+1 \leq j < k)) \nonumber \\
            \Leftrightarrow &\mathfrak{I}, i \models \mathfrak{a}_{\phi_{2}}(\phi_{2})\text{ or }(\mathfrak{I}, i \models \mathfrak{a}_{\phi_{1}}(\phi_{1})\text{ and }(i < n\text{ and } \\
            &\mathfrak{I}, i+1 \models \mathfrak{a}(\phi_{1} \mathsf{U}_{p-1} \phi_{2}))) \nonumber \\
            \Leftrightarrow &\mathfrak{I}, i \models \mathfrak{a}(\phi_{2} \vee (\phi_{1} \wedge \Circle (\phi_{1} \mathsf{U}_{p-1} \phi_{2})))
        \end{align}
        \eqref{eq:extension/metrictemporaloperators/proofProp3/eq9.3} is equivalent to \eqref{eq:extension/metrictemporaloperators/proofProp3/eq9.2} because
        \begin{itemize}
            \item in case $i < n$, either $\phi_{2}$ needs to be satisfied now, at time point $i$, or $\phi_{1}$ needs to be satisfied now, at time point $i$, and there needs to be a future time point $k$, $i+1 \leq k \leq \mathsf{min}((i+1)+(p-1),n)$,
            where $\phi_{2}$ is satisfied and $\phi_{1}$ is satisfied for all time points $j$, $i+1 \leq j < k$, for the query to be satisfied.
            \item in case $i = n$, $\phi_{2}$ needs to be satisfied now, at time point $i$, for the query to be satisfied.
            There are no future time points $k$, $n+1 \leq k \leq \mathsf{min}((i+1)+(p-1),n)$.
            Since $i = n$ is true, $\mathfrak{I}, i \models \mathfrak{a}_{\phi_{2}}(\phi_{2})$ is equivalent to there is $k$, $n \leq k \leq \mathsf{min}(i+p,n)$, with $\mathfrak{I}, k \models \mathfrak{a}_{\phi_{2}}(\phi_{2})$
            and $\mathfrak{I}, j \models \mathfrak{a}_{\phi_{1}}(\phi_{1})$ for all $j, n \leq j < k$, and $i < n$ is not true, thus the ``or''-statement does not affect satisfaction.
        \end{itemize}
    \end{itemize}
\end{proof}

The semantics of the MTOs can now be used to extend the algorithm specified in \cite{borgwardt2015temporalizing}.

The functions $\mathsf{eval}^{n}: \mathsf{AT}^{n}_{\phi} \rightarrow 2^{\Delta^{\mathsf{N}_\mathsf{V}}}, n \geq 0$ in \cite{borgwardt2015temporalizing} have then to be extended as follows:
\begin{table}[H]
    \centering
    \begin{tabular*}{\textwidth}{@{}ll@{}}
        \toprule
        $\alpha$                                                          & $\mathsf{eval}^{n}(\alpha)$                                                   \\ \midrule
        $[\ldots]$                & $[\ldots]$ \\
        $x^{\Circle_{p} \psi_{1}}_{j}$ with $j < n$                     & $\mathsf{Ans}(\Circle_{p-1} \psi_{1}, \mathfrak{I}^{(n)}, j+1)$                             \\
        $x^{\CIRCLE_{p} \psi_{1}}_{j}$ with $j < n$                     & $\mathsf{Ans}(\CIRCLE_{p-1} \psi_{1}, \mathfrak{I}^{(n)}, j+1)$                             \\
        $x^{\Box_{p} \psi_{1}}_{j}$ with $j < n$                        & $\mathsf{Ans}(\Box_{p-1} \psi_{1}, \mathfrak{I}^{(n)}, j+1)$              \\
        $x^{\Diamond_{p} \psi_{1}}_{j}$ with $j < n$                    & $\mathsf{Ans}(\Diamond_{p-1} \psi_{1}, \mathfrak{I}^{(n)}, j+1)$          \\
        $x^{\psi_{1} \mathsf{U}_{p} \psi_{2}}_{j}$ with $j < n$         & $\mathsf{Ans}(\psi_{1} \mathsf{U}_{p-1} \psi_{2}, \mathfrak{I}^{(n)}, j+1)$   \\
        $x^{\Circle_{p} \psi_{1}}_{n}$                                  & $\emptyset$                                                                   \\
        $x^{\CIRCLE_{p} \psi_{1}}_{n}$                                  & $\Delta^{\mathsf{N}_\mathsf{V}}$                                              \\
        $x^{\Box_{p} \psi_{1}}_{n}$                                     & $\Delta^{\mathsf{N}_\mathsf{V}}$                                              \\
        $x^{\Diamond_{p} \psi_{1}}_{n}$                                 & $\emptyset$                                                                   \\
        $x^{\psi_{1} \mathsf{U}_{p} \psi_{2}}_{n}$                      & $\emptyset$                                                                   \\ \bottomrule
    \end{tabular*}
    \caption{$\mathsf{eval}^{n}(\alpha)$ with metric temporal operators}
    \label{tab:extension/metrictemporaloperators/eval}
\end{table}

\begin{minipage}[H]{\textwidth}
    The function $\Phi_{0}(\psi): \mathsf{Sub}(\phi) \rightarrow \mathsf{AT}^{0}_{\phi}$ in \cite{borgwardt2015temporalizing} has to be extended as follows:
    \begin{table}[H]
        \centering
        \begin{tabular*}{\textwidth}{@{}ll@{}}
            \toprule
            $\psi$                                                & $\Phi_{0}(\psi)$                                                                              \\ \midrule
            $[\ldots]$                & $[\ldots]$ \\
            $\Circle_{p} \psi_{1}$ with $p > 0$                 & $x^{\Circle_{p} \psi_{1}}_{0}$                                                                \\
            $\CIRCLE_{p} \psi_{1}$ with $p > 0$                 & $x^{\CIRCLE_{p} \psi_{1}}_{0}$                                                                \\
            $\Circle^{-}_{p} \psi_{1}$ with $p > 0$             & $\emptyset$                                                                                   \\
            $\CIRCLE^{-}_{p} \psi_{1}$ with $p > 0$             & $\Delta^{\mathsf{N}_\mathsf{V}}$                                                              \\
            $\Box_{p} \psi_{1}$ with $p > 0$                    & $\Phi_{0}(\psi_{1}) \cap x^{\Box_{p} \psi_{1}}_{0}$                                         \\
            $\Box^{-}_{p} \psi_{1}$                             & $\Phi_{0}(\psi_{1})$                                                                          \\
            $\Diamond_{p} \psi_{1}$ with $p > 0$                & $\Phi_{0}(\psi_{1}) \cup x^{\Diamond_{p} \psi_{1}}_{0}$                                     \\
            $\Diamond^{-}_{p} \psi_{1}$                         & $\Phi_{0}(\psi_{1})$                                                                          \\
            $\psi_{1} \mathsf{U}_{p} \psi_{2}$ with $p > 0$     & $\Phi_{0}(\psi_{2}) \cup (\Phi_{0}(\psi_{1}) \cap x^{\psi_{1} \mathsf{U}_{p} \psi_{2}}_{0})$    \\
            $\psi_{1} \mathsf{S}_{p} \psi_{2}$                  & $\Phi_{0}(\psi_{2})$                                                                          \\
            $\Circle_{0} \psi_{1}$                              & $\Phi_{0}(\psi_{1})$                                                                          \\
            $\CIRCLE_{0} \psi_{1}$                              & $\Phi_{0}(\psi_{1})$                                                                          \\
            $\Circle^{-}_{0} \psi_{1}$                          & $\Phi_{0}(\psi_{1})$                                                                          \\
            $\CIRCLE^{-}_{0} \psi_{1}$                          & $\Phi_{0}(\psi_{1})$                                                                          \\
            $\Box_{0} \psi_{1}$                                 & $\Phi_{0}(\psi_{1})$                                                                          \\
            $\Diamond_{0} \psi_{1}$                             & $\Phi_{0}(\psi_{1})$                                                                          \\
            $\psi_{1} \mathsf{U}_{0} \psi_{2}$                  & $\Phi_{0}(\psi_{2})$                                                                          \\\bottomrule
        \end{tabular*}
        \caption{$\Phi_{0}(\psi)$ with metric temporal operators}
        \label{tab:extension/metrictemporaloperators/phi0}
    \end{table}
\end{minipage}

\begin{minipage}[H]{\textwidth}
    The function $\Phi^{0}_{i}(\psi): \mathsf{Sub}(\phi) \rightarrow \mathsf{AT}^{i}_{\phi},\ i>0$ in \cite{borgwardt2015temporalizing} has to be extended as follows:
    \begin{table}[H]
        \centering
        \begin{tabular*}{\textwidth}{@{}ll@{}}
            \toprule
            $\psi$                                                & $\Phi^{0}_{i}(\psi)$                                                                                          \\ \midrule
            $[\ldots]$                & $[\ldots]$ \\
            $\Circle_{p} \psi_{1}$ with $p > 0$                 & $x^{\Circle_{p} \psi_{1}}_{i}$                                                                                \\
            $\CIRCLE_{p} \psi_{1}$ with $p > 0$                 & $x^{\CIRCLE_{p} \psi_{1}}_{i}$                                                                                \\
            $\Circle^{-}_{p} \psi_{1}$ with $p > 0$             & $\Phi_{i-1}(\Circle^{-}_{p-1} \psi_{1})$                                                                                        \\
            $\CIRCLE^{-}_{p} \psi_{1}$ with $p > 0$             & $\Phi_{i-1}(\CIRCLE^{-}_{p-1} \psi_{1})$                                                                                        \\
            $\Box_{p} \psi_{1}$ with $p > 0$                    & $\Phi^{0}_{i}(\psi_{1}) \cap x^{\Box_{p} \psi_{1}}_{i}$                                                     \\
            $\Box^{-}_{p} \psi_{1}$ with $p > 0$                & $\Phi^{0}_{i}(\psi_{1}) \cap \Phi_{i-1}(\Box^{-}_{p-1} \psi_{1})$                                             \\
            $\Diamond_{p} \psi_{1}$ with $p > 0$                & $\Phi^{0}_{i}(\psi_{1}) \cup x^{\Diamond_{p} \psi_{1}}_{i}$                                                 \\
            $\Diamond^{-}_{p} \psi_{1}$ with $p > 0$            & $\Phi^{0}_{i}(\psi_{1}) \cup \Phi_{i-1}(\Diamond^{-}_{p-1} \psi_{1})$                                         \\
            $\psi_{1} \mathsf{U}_{p} \psi_{2}$ with $p > 0$     & $\Phi^{0}_{i}(\psi_{2}) \cup (\Phi^{0}_{i}(\psi_{1}) \cap x^{\psi_{1} \mathsf{U}_{p} \psi_{2}}_{i})$        \\
            $\psi_{1} \mathsf{S}_{p} \psi_{2}$ with $p > 0$     & $\Phi^{0}_{i}(\psi_{2}) \cup (\Phi^{0}_{i}(\psi_{1}) \cap \Phi_{i-1}(\psi_{1} \mathsf{S}_{p-1} \psi_{2}))$    \\
            $\Circle_{0} \psi_{1}$                              & $\Phi^{0}_{i}(\psi_{1})$                                                                                      \\
            $\CIRCLE_{0} \psi_{1}$                              & $\Phi^{0}_{i}(\psi_{1})$                                                                                      \\
            $\Circle^{-}_{0} \psi_{1}$                          & $\Phi^{0}_{i}(\psi_{1})$                                                                                      \\
            $\CIRCLE^{-}_{0} \psi_{1}$                          & $\Phi^{0}_{i}(\psi_{1})$                                                                                      \\
            $\Box_{0} \psi_{1}$                                 & $\Phi^{0}_{i}(\psi_{1})$                                                                                      \\
            $\Box^{-}_{0} \psi_{1}$                             & $\Phi^{0}_{i}(\psi_{1})$                                                                                      \\
            $\Diamond_{0} \psi_{1}$                             & $\Phi^{0}_{i}(\psi_{1})$                                                                                      \\
            $\Diamond^{-}_{0} \psi_{1}$                         & $\Phi^{0}_{i}(\psi_{1})$                                                                                      \\
            $\psi_{1} \mathsf{U}_{0} \psi_{2}$                  & $\Phi^{0}_{i}(\psi_{2})$                                                                                      \\
            $\psi_{1} \mathsf{S}_{0} \psi_{2}$                  & $\Phi^{0}_{i}(\psi_{2})$                                                                                      \\ \bottomrule
        \end{tabular*}
        \caption{$\Phi^{0}_{i}(\psi)$ with metric temporal operators}
        \label{tab:extension/metrictemporaloperators/phiI}
    \end{table}
\end{minipage}
\\

$\mathsf{Sub}(\phi)$ now includes all queries $\Circle_{m} \psi_{1}$, $\CIRCLE_{m} \psi_{1}$, $\Circle^{-}_{m} \psi_{1}$, $\CIRCLE^{-}_{m} \psi_{1}$,
$\Box_{m} \psi_{1}$, $\Box^{-}_{m} \psi_{1}$, $\Diamond_{m} \psi_{1}$, $\Diamond^{-}_{m} \psi_{1}$, $\psi_{1} \mathsf{U}_{m} \psi_{2}$ and $\psi_{1} \mathsf{S}_{m} \psi_{2}$
with $m$, $0 \leq m \leq p$.
$\mathsf{FSub}(\phi)$, the subset of queries from $\mathsf{Sub}(\phi)$ that start with a future operator, is extended accordingly.
It is noteworthy, that even though the sets denoted by $\mathsf{Sub}(\phi)$ and $\mathsf{FSub}(\phi)$ for $\phi$ with MTOs
are different from the sets for $\phi$ without MTOs, they are the same size, e.g.\ for Query \ref{qu:queries/investigationqueries/practical/qu3}
without MTOs $\mathsf{Sub}(\phi) = \{\phi,\allowbreak \Circle^{-}\Circle^{-}\Circle^{-}\Circle^{-}\Circle^{-} \psi,\allowbreak
\Circle^{-}\Circle^{-}\Circle^{-}\Circle^{-} \psi,\allowbreak \Circle^{-}\Circle^{-}\Circle^{-} \psi,\allowbreak \Circle^{-}\Circle^{-} \psi,\allowbreak \Circle^{-} \psi, \psi\}$
and with MTOs $\mathsf{Sub}(\phi) = \{\phi,\allowbreak \Diamond^{-}_{5} \psi,\allowbreak \Diamond^{-}_{4} \psi,\allowbreak \Diamond^{-}_{3} \psi,\allowbreak \Diamond^{-}_{2} \psi,\allowbreak \Diamond^{-}_{1} \psi,\allowbreak \Diamond^{-}_{0} \psi\}$.

\begin{theorem}
    \label{th:extension/metrictemporaloperators/corretandbounded}
    Extending the algorithm from \cite{borgwardt2015temporalizing} by metric temporal operators preserves correctness and boundedness.
\end{theorem}

\begin{proof}
    To prove that the correctness and boundedness of the algorithm is preserved, the necessary cases are added to the
    corresponding proofs from \cite{borgwardt2015temporalizing}.

    \begin{restatable}[ct.\ Lemma 6.3 in \cite{borgwardt2015temporalizing}]{lemma}{lemmaMTOsPhiZero}
        \label{lem:extension/metrictemporaloperators/lemma5}
        The function $\Phi_{0}$ with metric temporal operators is correct for 0.
    \end{restatable}

    \begin{proof}
        It is shown by induction on the structure of the subqueries $\psi \in \mathsf{Sub}(\phi)$ that $\mathsf{eval}^{n}(\Phi_{0}(\psi))$
        is equal to $\mathsf{Ans}(\psi, \mathfrak{I}^{(n)},0)$ for all $n \geq 0$.
        The missing cases can be found in Appendix \ref{ch:appendixC}.\\
        If $\psi = \Circle_{0} \psi_{1}$, $\psi = \CIRCLE_{0} \psi_{1}$, $\psi = \Circle^{-}_{0} \psi_{1}$, $\psi = \CIRCLE^{-}_{0} \psi_{1}$, $\psi = \Box_{0} \psi_{1}$, $\psi = \Box^{-}_{0} \psi_{1}$, $\psi = \Diamond_{0} \psi_{1}$ or $\psi = \Diamond^{-}_{0} \psi_{1}$, then
        \[\mathsf{eval}^{n}(\Phi_{0}(\psi)) = \mathsf{eval}^{n}(\Phi_{0}(\psi_{1})).\]
        This is by induction equal to $Ans(\psi_{1}, \mathfrak{I}^{(n)},0)$ which then is, as shown in Proposition \ref{prop:extension/metrictemporaloperators/prop2},
        equal to $\mathsf{Ans}(\psi, \mathfrak{I}^{(n)},0)$.\\
        If $\psi = \psi_{1} \mathsf{U}_{0} \psi_{2}$ or $\psi = \psi_{1} \mathsf{S}_{0} \psi_{2}$, then
        \[\mathsf{eval}^{n}(\Phi_{0}(\psi)) = \mathsf{eval}^{n}(\Phi_{0}(\psi_{2})).\]
        This is by induction equal to $Ans(\psi_{2}, \mathfrak{I}^{(n)},0)$ which then is, as shown in Proposition \ref{prop:extension/metrictemporaloperators/prop2},
        equal to $\mathsf{Ans}(\psi, \mathfrak{I}^{(n)},0)$.\\
        If $\psi = \Circle^{-}_{p} \psi_{1}$, $\psi = \CIRCLE^{-}_{p} \psi_{1}$, $\psi = \Box^{-}_{p} \psi_{1}$ or $\psi = \Diamond^{-}_{p} \psi_{1}$ and $p > 0$, then
        \[\mathsf{eval}^{n}(\Phi_{0}(\psi)) = \mathsf{eval}^{n}(\Phi_{0}(\psi_{1})).\]
        This is by induction equal to $\mathsf{Ans}(\psi_{1}, \mathfrak{I}^{(n)},0)$ which then is, as shown in Proposition \ref{prop:extension/metrictemporaloperators/prop3},
        equal to $\mathsf{Ans}(\psi, \mathfrak{I}^{(n)},0)$.\\
        If $\psi = \Circle_{p} \psi_{1}$ and $p > 0$, then
        \begin{equation}
            \notag
            \label{eq:extension/metrictemporaloperators/proofLemma5/eq1}
            \begin{split}
                \mathsf{eval}^{n}(\Phi_{0}(\psi)) &= \mathsf{eval}^{n}(x^{\Circle_{p} \psi_{1}}_{0}) \\
                &= \left\{\begin{array}{lr}
                              \mathsf{Ans}(\Circle_{p-1} \psi_{1}, \mathfrak{I}^{(n)},1) & \text{if } n > 0\\
                              \emptyset      & \text{if } n = 0\\
                \end{array}\right\} \\
                &= \mathsf{Ans}(\psi, \mathfrak{I}^{(n)},0)
            \end{split}
        \end{equation}
        If $\psi = \Box_{p} \psi_{1}$ and $p > 0$, then
        \begin{equation}
            \notag
            \label{eq:extension/metrictemporaloperators/proofLemma5/eq3}
            \begin{split}
                \mathsf{eval}^{n}(\Phi_{0}(\psi)) &= \mathsf{eval}^{n}(\Phi_{0}(\psi_{1})) \cap \mathsf{eval}^{n}(x^{\Box_{p} \psi_{1}}_{0}) \\
                &= \mathsf{Ans}(\psi_{1}, \mathfrak{I}^{(n)},0) \cap \left\{\begin{array}{lr}
                                                                                \mathsf{Ans}(\Box_{p-1} \psi_{1}, \mathfrak{I}^{(n)},1) & \text{if } n > 0\\
                                                                                \Delta^{\mathsf{N}_\mathsf{V}}       & \text{if } n = 0\\
                \end{array}\right\} \\
                &= \mathsf{Ans}(\psi, \mathfrak{I}^{(n)},0)
            \end{split}
        \end{equation}
        If $\psi = \psi_{1} \mathsf{U}_{p} \psi_{2}$ and $p > 0$, then
        \begin{equation}
            \notag
            \label{eq:extension/metrictemporaloperators/proofLemma5/eq5}
            \begin{split}
                \mathsf{eval}^{n}(\Phi_{0}(\psi)) &= \mathsf{eval}^{n}(\Phi_{0}(\psi_{2})) \cup (\mathsf{eval}^{n}(\Phi_{0}(\psi_{1})) \cap \mathsf{eval}^{n}(x^{\psi_{1} \mathsf{U}_{p} \psi_{2}}_{0})) \\
                &= \mathsf{Ans}(\psi_{2}, \mathfrak{I}^{(n)},0)\cup(\mathsf{Ans}(\psi_{1}, \mathfrak{I}^{(n)},0) \cap \left\{\begin{array}{lr}
                                                                                                                                 \mathsf{Ans}(\psi_{1} \mathsf{U}_{p-1} \psi_{2}, \mathfrak{I}^{(n)},1) & \text{if } n > 0\\
                                                                                                                                 \emptyset       & \text{if } n = 0\\
                \end{array}\right\}) \\
                &= \mathsf{Ans}(\psi, \mathfrak{I}^{(n)},0)
            \end{split}
        \end{equation}
    \end{proof}

    \begin{restatable}[ct.\ Lemma 6.4 in \cite{borgwardt2015temporalizing}]{lemma}{lemmaMTOsPhiiminus}
        \label{lem:extension/metrictemporaloperators/lemma6}
        If $\Phi_{i-1}$ with metric temporal operators is correct for i-1, then $\Phi^{0}_{i}$ with metric temporal operators is correct for i.
    \end{restatable}

    \begin{proof}
        It is shown by induction on the structure of the subqueries $\psi \in \mathsf{Sub}(\phi)$ that $\mathsf{eval}^{n}(\Phi^{0}_{i}(\psi))$
        is equal to $\mathsf{Ans}(\psi, \mathfrak{I}^{(n)},i)$ for all $n \geq i$.
        The missing cases can be found in Appendix \ref{ch:appendixC}.\\

        If $\psi = \Circle_{0} \psi_{1}$, $\psi = \CIRCLE_{0} \psi_{1}$, $\psi = \Circle^{-}_{0} \psi_{1}$, $\psi = \CIRCLE^{-}_{0} \psi_{1}$, $\psi = \Box_{0} \psi_{1}$, $\psi = \Box^{-}_{0} \psi_{1}$, $\psi = \Diamond_{0} \psi_{1}$ or $\psi = \Diamond^{-}_{0} \psi_{1}$, then
        \begin{equation}
            \notag
            \label{eq:extension/metrictemporaloperators/proofLemma6/eq1}
            \begin{split}
                \mathsf{eval}^{n}(\Phi^{0}_{i}(\psi)) &= \mathsf{eval}^{n}(\Phi^{0}_{i}(\psi_{1})) \\
                &= \mathsf{Ans}(\psi_{1}, \mathfrak{I}^{(n)},i) \\
                &= \mathsf{Ans}(\psi, \mathfrak{I}^{(n)},i)
            \end{split}
        \end{equation}
        If $\psi = \psi_{1} \mathsf{U}_{0} \psi_{2}$ or $\psi = \psi_{1} \mathsf{S}_{0} \psi_{2}$, then
        \begin{equation}
            \notag
            \label{eq:extension/metrictemporaloperators/proofLemma6/eq2}
            \begin{split}
                \mathsf{eval}^{n}(\Phi^{0}_{i}(\psi)) &= \mathsf{eval}^{n}(\Phi^{0}_{i}(\psi_{2})) \\
                &= \mathsf{Ans}(\psi_{2}, \mathfrak{I}^{(n)},i) \\
                &= \mathsf{Ans}(\psi, \mathfrak{I}^{(n)},i)
            \end{split}
        \end{equation}
        If $\psi = \Circle_{p} \psi_{1}$ and $p > 0$, then
        \begin{equation}
            \notag
            \label{eq:extension/metrictemporaloperators/proofLemma6/eq3}
            \begin{split}
                \mathsf{eval}^{n}(\Phi^{0}_{i}(\psi)) &= \mathsf{eval}^{n}(x^{\Circle_{p} \psi_{1}}_{i}) \\
                &= \left\{\begin{array}{lr}
                              \mathsf{Ans}(\Circle_{p-1} \psi_{1}, \mathfrak{I}^{(n)},i+1) & \text{if } n > i\\
                              \emptyset      & \text{if } n = i\\
                \end{array}\right\} \\
                &= \mathsf{Ans}(\psi, \mathfrak{I}^{(n)},i)
            \end{split}
        \end{equation}
        If $\psi = \Box^{-}_{p} \psi_{1}$ and $p > 0$, then
        \begin{equation}
            \notag
            \label{eq:extension/metrictemporaloperators/proofLemma6/eq8}
            \begin{split}
                \mathsf{eval}^{n}(\Phi^{0}_{i}(\psi)) &= \mathsf{eval}^{n}(\Phi^{0}_{i}(\psi_{1})) \cap \mathsf{eval}^{n}(\Phi_{i-1}(\Box^{-}_{p} \psi_{1})) \\
                &= \mathsf{Ans}(\psi_{1}, \mathfrak{I}^{(n)},i) \cap \mathsf{Ans}(\Box^{-}_{p-1} \psi_{1}, \mathfrak{I}^{(n)},i-1) \\
                &= \mathsf{Ans}(\psi, \mathfrak{I}^{(n)},i)
            \end{split}
        \end{equation}
        If $\psi = \psi_{1} \mathsf{U}_{p} \psi_{2}$ and $p > 0$, then
        \begin{equation}
            \notag
            \label{eq:extension/metrictemporaloperators/proofLemma6/eq11}
            \begin{split}
                \mathsf{eval}^{n}(\Phi^{0}_{i}(\psi)) &= \mathsf{eval}^{n}(\Phi^{0}_{i}(\psi_{2})) \cup (\mathsf{eval}^{n}(\Phi^{0}_{i}(\psi_{1})) \cap \mathsf{eval}^{n}(x^{\psi_{1} \mathsf{U}_{p} \psi_{2}}_{i})) \\
                &= \mathsf{Ans}(\psi_{2}, \mathfrak{I}^{(n)},i)\cup(\mathsf{Ans}(\psi_{1}, \mathfrak{I}^{(n)},i) \cap \left\{\begin{array}{lr}
                                                                                                                                 \mathsf{Ans}(\psi_{1} \mathsf{U}_{p-1} \psi_{2}, \mathfrak{I}^{(n)},i+1) & \text{if } n > i\\
                                                                                                                                 \emptyset       & \text{if } n = i\\
                \end{array}\right\}) \\
                &= \mathsf{Ans}(\psi, \mathfrak{I}^{(n)},i)
            \end{split}
        \end{equation}
    \end{proof}

    \begin{lemma}[ct.\ Lemma 6.5 in \cite{borgwardt2015temporalizing}]
        \label{lem:extension/metrictemporaloperators/lemma7}
        If $\Phi_{i-1}$ with metric temporal operators is correct for i-1 and (i-1)-bounded, then we can construct a function $\Phi_{i}: \mathsf{Sub}(\phi)
        \rightarrow \mathsf{AT}^{i}_{\phi}$ with metric temporal operators that is correct for i and i-bounded.
    \end{lemma}

    \begin{proof}
        The function $\mathsf{update}(x^{\psi^{j}}_{i-1})$, introduced in \cite{borgwardt2015temporalizing}, needs to be extended, before it then can be shown for all $n \geq i$ that
        $\mathsf{eval}^{n}(x^{\psi^{j}}_{i-1})$ is still equal to $\mathsf{eval}^{n}(\mathsf{update}(x^{\psi^{j}}_{i-1}))$.
        For operators with metric temporal operators, $\mathsf{update}(x^{\psi^{j}}_{i-1})$ looks like this:
        \begin{equation}
            \notag
            \label{eq:extension/metrictemporaloperators/proofLemma7/eq1}
            \begin{split}
                \mathsf{update}(x^{\psi^{j}}_{i-1}) := \left\{\begin{array}{lr}
                                                                    \Phi^{j-1}_{i}(\Circle_{p-1} \psi_{1})  & \text{if } \psi^{j}=\Circle_{p} \psi_{1} \\
                                                                    \Phi^{j-1}_{i}(\CIRCLE_{p-1} \psi_{1})  & \text{if } \psi^{j}=\CIRCLE_{p} \psi_{1} \\
                                                                    \Phi^{j-1}_{i}(\Box_{p-1} \psi_{1})  & \text{if } \psi^{j}=\Box_{p} \psi_{1} \\
                                                                    \Phi^{j-1}_{i}(\Diamond_{p-1} \psi_{1})  & \text{if } \psi^{j}=\Diamond_{p} \psi_{1} \\
                                                                    \Phi^{j-1}_{i}(\psi_{1} \mathsf{U}_{p-1} \psi_{2})  & \text{if } \psi^{j}=\psi_{1} \mathsf{U}_{p} \psi_{2} \\
                \end{array}\right\}
            \end{split}
        \end{equation}
        By definition \begin{gather*}
            \mathsf{eval}^{n}(x^{\Circle_{p} \psi_{1}}_{i-1}) = \mathsf{Ans}(\Circle_{p-1} \psi_{1}, \mathfrak{I}^{(n)},i),\\
            \mathsf{eval}^{n}(x^{\CIRCLE_{p} \psi_{1}}_{i-1}) = \mathsf{Ans}(\CIRCLE_{p-1} \psi_{1}, \mathfrak{I}^{(n)},i),\\
            \mathsf{eval}^{n}(x^{\Box_{p} \psi_{1}}_{i-1}) = \mathsf{Ans}(\Box_{p-1} \psi_{1}, \mathfrak{I}^{(n)},i),\\
            \mathsf{eval}^{n}(x^{\Diamond_{p} \psi_{1}}_{i-1}) = \mathsf{Ans}(\Diamond_{p-1} \psi_{1}, \mathfrak{I}^{(n)},i)\text{ and}\\
            \mathsf{eval}^{n}(x^{\psi_{1} \mathsf{U}_{p} \psi_{2}}_{i-1}) = \mathsf{Ans}(\psi_{1} \mathsf{U}_{p-1} \psi_{2}, \mathfrak{I}^{(n)},i).
        \end{gather*}
        Since $\Phi^{j-1}_{i}$ is correct for $i$, these are the same sets as \[\mathsf{eval}^{n}(\Phi^{j-1}_{i}(\psi^{j}))=\mathsf{eval}^{n}(\mathsf{update}(x^{\psi^{j}}_{i-1})).\]

        It remains to show $i$-boundedness of $\Phi_{i}=\Phi^{k}_{i}$.
        In \cite{borgwardt2015temporalizing} this is again proven by induction on $j$.
        It therefore suffices to add the missing cases.
        It is enough to show that $\mathsf{update}(x^{\psi^{j}}_{i-1})$ contains only variables from $\mathsf{Var}^{\psi^{j}}_{i}$.
        Since $\Phi^{j-1}_{i}$ differs from $\Phi^{0}_{i}$ only in the replacement of some variables with index $i-1$,
        \begin{align}
            &\Phi^{j-1}_{i}(\Circle_{p-1} \psi_{1})  = x^{\Circle_{p} \psi_{1}}_{i} \nonumber\\
            &\Phi^{j-1}_{i}(\CIRCLE_{p-1} \psi_{1})  = x^{\CIRCLE_{p} \psi_{1}}_{i} \nonumber\\
            &\Phi^{j-1}_{i}(\Box_{p-1} \psi_{1})  = \Phi^{j-1}_{i}(\psi_{1}) \cap x^{\Box_{p} \psi_{1}}_{i} \nonumber\\
            &\Phi^{j-1}_{i}(\Diamond_{p-1} \psi_{1})  = \Phi^{j-1}_{i}(\psi_{1}) \cup x^{\Diamond_{p} \psi_{1}}_{i} \nonumber\\
            &\Phi^{j-1}_{i}(\psi_{1} \mathsf{U}_{p-1} \psi_{2})  =\Phi^{j-1}_{i}(\psi_{2}) \cup (\Phi^{j-1}_{i}(\psi_{1}) \cap x^{\psi_{1} \mathsf{U}_{p} \psi_{2}}_{i}) \nonumber
        \end{align}

        By the induction hypothesis each $\Phi^{j-1}_{i}(\psi_{m}), m=1,2$, contains only variables from $\mathsf{Var}^{\psi_{m}}_{i} = \mathsf{Var}^{\psi^{j}}_{i} \setminus \{x^{\psi^{j}}_{i}\}$
        and $\mathsf{Var}^{\psi_{m}}_{i-1} \cap \{x^{\psi^{j}}_{i-1},\dots,x^{\psi^{k}}_{i-1}\}$.
        Since every variable $x^{\psi^{'}}_{i-1} \in \mathsf{Var}^{\psi_{1}}_{i-1}$ must satisfy $\psi^{'} \in \mathsf{FSub}(\psi_{1})$ the second set $\mathsf{Var}^{\psi_{1}}_{i-1} \cap \{x^{\psi^{j}}_{i-1},...,x^{\psi^{k}}_{i-1}\}$
        is empty.
        This follows from the total order $\psi^{1} \prec$ \dots $\prec \psi^{k}$ on the set $\mathsf{FSub}(\phi) = \{\psi^{1} , \dots, \psi^{k}\}$ presented in \cite{borgwardt2015temporalizing},
        i.e. $\psi^{'} \in \mathsf{FSub}(\psi^{j}) \setminus \{\psi^{j}\}$, and thus $\psi^{'} \prec \psi^{j}$.
    \end{proof}

    This concludes the proof of Theorem \ref{th:extension/metrictemporaloperators/corretandbounded}.
\end{proof}


\section{Combining Extensions}
\label{sec:extension/combiningExtensions}
To decide whether extending the implementation of the bounded history encoding from \cite{borgwardt2015temporalizing} is reasonable,
the
In the following, only the extensions of the bounded history encoding from \cite{borgwardt2015temporalizing} by a filter operator and metric temporal operators will be discussed,
as $\Box, \Box^{-}, \Diamond$ and $\Diamond^{-}$ can be seen as special cases of metric temporal operators, where $p = n-i$ or $p = i$, respectively.

From the proofs of Theorem REF THEOREM XY and Theorem REF THEOREM YZ it follows, that extending the bounded history encoding from \cite{borgwardt2015temporalizing} by a filter operator and metric temporal operators preserves correctness and boundedness.

In terms of expressiveness, only the filter operator adds new expressiveness.
This has already been hinted at in Chapter REF CHAPTER 2, when introducing the metric temporal operators and follows from Proposition REF PROPOSITION XY and Proposition REF PROPOSITION YZ.
Since the filter operator is defined to be an SQL statement, the expressiveness added by introducing the filter operator is the same as an SQL statement on a single table.

However, the metric temporal operators are still relevant, as they can reduce the size of queries significantly.
A query without metric temporal operators that is equivalent to, e.g., $\Box$ with a time limit $t$, consists of at least $2*t$ operators, a $\wedge$ and a $\CIRCLE$ for each time point up to the limit.
Whereas, the corresponding query with metric temporal operators consists of only one operator $\Box_{t}$ regardless of the size of $t$.

Lastly, it is feasible to implement the extensions as the main modification needed are case-distinctions in the implementations of $\Phi_{0}(\psi)$ and $\Phi^{0}_{i}(\psi)$.

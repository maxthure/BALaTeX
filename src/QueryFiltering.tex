Second, the algorithm presented in \cite{borgwardt2015temporalizing} is extended by a filter operator.
This filter operator $f[\phi]$ denotes that the filter query $f$ is applied to the result of $\phi$.
Since $\mathcal{Q}$-queries are not further specified, it is impossible to determine at which position in $f$ the result of $\phi$ should be inserted.
However, it is known that the implementation of the BHE \cite{boundedhistoryencodingalgorithm} is based on SQL.
Therefore it can be defined that $f$ is an SQL statement and the result of $\phi$ is inserted at ``SELECT * FROM $\phi$''.

The semantics of TQs from Definition \ref{def:Extension/operatorsnotimplemented/semantics} are extended as follows:
\begin{definition}[semantics of temporal queries with a filter operator ct.\ Definition 3.3 in \cite{borgwardt2015temporalizing}]
    \label{def:extension/filtering/semantics}
    Let $\phi$ be a TQ, $\mathfrak{I} = (I_{i})_{0 \leq i \leq n}$ a sequence of interpretations over a common domain,
    $\mathfrak{a}:\mathsf{FVar}(\phi) \rightarrow \mathsf{N}_{\mathsf{C}}$ a variable assignment, $f$ be an SQL statement, and $i$ be an integer with $0 \leq i \leq n$.
    The \textit{satisfaction relation} $\mathfrak{I}, i \models \mathfrak{a}(\phi)$ is defined by induction on the structure of $\phi$ as follows:
    \begin{table}[H]
        \centering
        \begin{tabular*}{\textwidth}{@{}ll@{}}
            \toprule
            $\phi$                            & $\mathfrak{I}, i \models \mathfrak{a}(\phi)$ iff  \\ \midrule
            $[\ldots]$                & $[\ldots]$ \\
            $f[\phi]$                       & $I_{i} \models \mathfrak{a}(f)$ and $\mathfrak{I},i \models \mathfrak{a}(\phi)$            \\ \bottomrule
        \end{tabular*}
        \caption{Satisfaction relation of temporal queries with a filter operator}
        \label{tab:extension/filtering/satisfactionrelation}
    \end{table}
\end{definition}
Now Query \ref{qu:queries/investigationqueries/practical/qu2} can be rewritten as follows:
\begin{align}
    &\text{``SELECT * FROM $(\Circle^{-}$(SELECT $marke$, AVG($price$) AS $price1$ FROM $autos$) $\wedge$}  \nonumber \\
    &\text{SELECT $marke$, AVG($price$) AS $price2$ FROM $autos$) WHERE $price1 < price2$''}.  \nonumber
\end{align}

The semantics of the filter operator can now be used to extend the algorithm specified in \cite{borgwardt2015temporalizing}.
The functions $\mathsf{eval}^{n}: \mathsf{AT}^{n}_{\phi} \rightarrow 2^{\Delta^{\mathsf{N}_\mathsf{V}}}, n \geq 0$ in \cite{borgwardt2015temporalizing} do not have to be extended.

The function $\Phi_{0}(\psi): \mathsf{Sub}(\phi) \rightarrow \mathsf{AT}^{0}_{\phi}$ in \cite{borgwardt2015temporalizing} has to be extended as follows:
\begin{table}[H]
    \centering
    \begin{tabular*}{\textwidth}{@{}ll@{}}
        \toprule
        $\psi$            & $\Phi_{0}(\psi)$                                      \\ \midrule
        $[\ldots]$                & $[\ldots]$ \\
        $f[\psi_{1}]$      & $\mathsf{Ans}(f[\Phi_{0}(\psi_{1})], I_{0})$     \\\bottomrule
    \end{tabular*}
    \caption{$\Phi_{0}(\psi)$ with a filter operator}
    \label{tab:extension/filtering/phi0}
\end{table}

The function $\Phi^{0}_{i}(\psi): \mathsf{Sub}(\phi) \rightarrow \mathsf{AT}^{i}_{\phi},\ i>0$ in \cite{borgwardt2015temporalizing} has to be extended as follows:
\begin{table}[H]
    \centering
    \begin{tabular*}{\textwidth}{@{}ll@{}}
        \toprule
        $\psi$            & $\Phi_{0}(\psi)$                                      \\ \midrule
        $[\ldots]$                & $[\ldots]$ \\
        $f[\psi_{1}]$      & $\mathsf{Ans}(f[\Phi^{0}_{i}(\psi_{1})], I_{i})$     \\\bottomrule
    \end{tabular*}
    \caption{$\Phi^{0}_{i}(\psi)$ with a filter operator}
    \label{tab:extension/filtering/phiI}
\end{table}

\begin{theorem}
    \label{th:extension/filtering/corretandbounded}
    Extending the algorithm from \cite{borgwardt2015temporalizing} by a filter operator preserves correctness and boundedness.
\end{theorem}

\begin{proof}
    To prove that the correctness and boundedness of the algorithm is preserved, the necessary cases are added to the
    corresponding proofs from \cite{borgwardt2015temporalizing}.

    \begin{lemma}[ct.\ Lemma 6.3 in \cite{borgwardt2015temporalizing}]
        \label{lem:extension/filtering/lemma5}
        The function $\Phi_{0}$ with a filter operator is correct for 0.
    \end{lemma}

    \begin{proof}
        It is shown by induction on the structure of the subqueries $\psi \in \mathsf{Sub}(\phi)$ that $\mathsf{eval}^{n}(\Phi_{0}(\psi))$
        is equal to $\mathsf{Ans}(\psi, \mathfrak{I}^{(n)},0)$ for all $n \geq 0$. \\
        If $\psi = f[\psi_{1}]$, then
        \[\mathsf{eval}^{n}(\Phi_{0}(\psi)) = \mathsf{Ans}(f[\Phi_{0}(\psi_{1})], I_{0}) = \mathsf{Ans}(\psi, \mathfrak{I}^{(n)},0).\]
    \end{proof}

    \begin{lemma}[ct.\ Lemma 6.4 in \cite{borgwardt2015temporalizing}]
        \label{lem:extension/filtering/lemma6}
        If $\Phi_{i-1}$ with a filter operator is correct for i-1, then $\Phi^{0}_{i}$ with a filter operator is correct for i.
    \end{lemma}

    \begin{proof}
        It is shown by induction on the structure of the subqueries $\psi \in \mathsf{Sub}(\phi)$ that $\mathsf{eval}^{n}(\Phi^{0}_{i}(\psi))$
        is equal to $\mathsf{Ans}(\psi, \mathfrak{I}^{(n)},i)$ for all $n \geq i$. \\
        If $\psi = f[\psi_{1}]$, then
        \[\mathsf{eval}^{n}(\Phi^{0}_{i}(\psi)) = \mathsf{Ans}(f[\Phi^{0}_{i}(\psi_{1})], I_{i}) = \mathsf{Ans}(\psi, \mathfrak{I}^{(n)},i).\]
    \end{proof}

    \begin{lemma}[ct.\ Lemma 6.5 in \cite{borgwardt2015temporalizing}]
        \label{lem:extension/filtering/lemma7}
        If $\Phi_{i-1}$ with a filter operator is correct for i-1 and (i-1)-bounded, then we can construct a function $\Phi_{i}: \mathsf{Sub}(\phi)
        \rightarrow \mathsf{AT}^{i}_{\phi}$ with a filter operator that is correct for i and i-bounded.
    \end{lemma}

    \begin{proof}
        Since $f[\psi_{1}]$ introduces no new variables, this follows directly from Lemma 6.5 in \cite{borgwardt2015temporalizing}.
    \end{proof}

    This concludes the proof of Theorem \ref{th:extension/filtering/corretandbounded}.
\end{proof}

\chapter{Queries ACHTUNG NICHT FERTIG}
\label{ch:queries}

After the last chapter covered in detail how the data stream is processed, this chapter takes a closer look at the queries
used for the investigation of the implementation of the bounded history encoding from \cite{borgwardt2015temporalizing}.
It further discusses whether the queries specified here can be expressed in the language of \cite{borgwardt2015temporalizing}.

The language of \cite{borgwardt2015temporalizing} consists of temporal queries (TQs).
They can be based on any atemporal query language $\mathcal{Q}$.
\begin{definition}[temporal query ct.\ Definition 3.2 in \cite{borgwardt2015temporalizing}]
    \label{def:queries/temporal}
    Given a query language $\mathcal{Q}$, $temporal\ \mathcal{Q}\text{-}queries$ are built from $\mathcal{Q}$-queries as follows:
    \begin{itemize}
        \item every $\mathcal{Q}$-query $\psi$ is a temporal $\mathcal{Q}$-query; and
        \item if $\phi_{1}$ and $\phi_{2}$ are temporal $\mathcal{Q}$-queries, then so are:
        \begin{itemize}
            \item $\phi_{1} \wedge \phi_{2}$ (conjunction), $\phi_{1} \vee \phi_{2}$ (disjunction),
            \item $\Circle \phi_{1}$ (strong next), $\CIRCLE \phi_{1}$ (weak next),
            \item $\Circle^{-} \phi_{1}$ (strong previous), $\CIRCLE^{-} \phi_{1}$ (weak previous),
            \item $\Box \phi_{1}$ (always), $\Box^{-} \phi_{1}$ (always in the past),
            \item $\Diamond \phi_{1}$ (eventually), $\Diamond^{-} \phi_{1}$ (some time in the past),
            \item $\phi_{1} \mathsf{U} \phi_{2}$ (until), and $\phi_{1} \mathsf{S} \phi_{2}$ (since).
        \end{itemize}
    \end{itemize}
\end{definition}

Since the implementation of the bounded history encoding from \cite{borgwardt2015temporalizing} uses a SQLite database, the atemporal query language $\mathcal{Q}$ in this thesis is SQL.


\section{Queries used for the investigation}
\label{sec:queries/investigationqueries}

In this thesis two different types of TQs are distinguished.
First, practical and meaningful TQs (PTQs) and second, randomized TQs (RTQs) based on the idea presented in \cite{schneider2019formally}.

\subsection{Practical TQs}
\label{subsec:queries/investigationqueries/practical}
PTQs show whether the bounded history encoding from \cite{borgwardt2015temporalizing} can provide relevant answers to queries.
This is necessary to find out if it has practical relevance and can be used, e.g. for market observation as it is done in this thesis.
For reasons of readability, only the following selected queries are discussed as examples:
\begin{align}
    &\text{"Which cars cost less than 10,000 at the last point in time} \label{qu:queries/investigationqueries/practical/qu1} \\
    &\text{and cost more than 10,000 at this point in time?"} \nonumber
\end{align}
\begin{align}
    &\text{"Which brands (or models) have an increased average price} \label{qu:queries/investigationqueries/practical/qu2} \\
    &\text{compared to the last point in time?"} \nonumber
\end{align}
\begin{align}
    &\text{"Which cars cost more than 10.000 some time in the last 6} \label{qu:queries/investigationqueries/practical/qu3}\\
    &\text{points in time?"} \nonumber
\end{align}

A COMPLETE LIST OF PTQs CAN BE FOUND IN THE APPENDIX REF APPENDIX XY.

\subsection{Randomized TQs}
\label{subsec:queries/investigationqueries/random}
RTQs are used to verify the impact of the queries on the overall research outcome.
They are necessary because the practical TQs might not be sufficiently diverse to provide reliable results.
RTQs can be constructed in any size.

\begin{definition}[randomized temporal query ct. \cite{schneider2019formally}]
    \label{def:queries/investigationqueries/random/randomizedtemporalqueries}
    Let $n \in \mathbb{N} \setminus \{0\}$.
    Let $\mathsf{Ops}$ be a set of available operators.
    A $randomized\ TQ$ of size $n$ can be constructed by recursively constructing its subqueries as follows:
    \begin{algorithm}[H]
        \caption*{$\mathsf{randomTQ}(n)$}
        \label{alg:queries/investigationqueries/random/algorithm}
        \algsetup{linenodelimiter=}
        \renewcommand{\algorithmiccomment}[1]{// #1}
        \begin{algorithmic}[1]
            \IF{$n = 1$}
            \RETURN $\mathcal{Q}-\text{query } \psi$
            \ELSE
            \STATE select a random operator $op$
            \IF{$op$ is unary}
            \RETURN $op(\mathsf{randomTQ}(n-1))$
            \ELSE[$op$ is binary]
            \STATE select a random $m \in \mathbb{N} \setminus \{0\}, m < n-1$
            \RETURN $(\mathsf{randomTQ}(m)\ op\ \mathsf{randomTQ}(n-m-1))$
            \ENDIF
            \ENDIF
        \end{algorithmic}
    \end{algorithm}
\end{definition}

A COMPLETE LIST OF GENERATED RTQs CAN BE FOUND IN THE APPENDIX REF APPENDIX XY.


\section{Are these queries expressible in the language from \cite{borgwardt2015temporalizing}?}
\label{sec:queries/expressibility}

\subsection{Practical TQs}
\label{subsec:queries/expressibility/practical}
As mentioned in CHAPTER REF CHAPTER 1, "autos" is the table in which all offers are stored.
The attribute "url" is distinct for each dataset and serves as an identifier for a car.
Hence, if the query asks for cars, it will actually return a set of urls.
This said, query REF QUERY1 can be translated into the language of \cite{borgwardt2015temporalizing}
as follows:
\begin{itemize}
    \item "[\dots] and [\dots]" indicates the "conjunction" operator from \cite{borgwardt2015temporalizing}
    \item "[\dots] at the last point in time [\dots]" indicates either a "strong previous" or a "weak previous" operator from \cite{borgwardt2015temporalizing}
    \item "Which cars cost less than 10,000 [\dots]" indicates the SQL-query "SELECT $url$ FROM $autos$ WHERE $price$ $<$ 10000"
    \item "[\dots] more than 10,000 [\dots]" indicates the SQL-query "SELECT $url$ FROM $autos$ WHERE $price$ $>$ 10000"
\end{itemize}
Thus, a resulting query in the language from \cite{borgwardt2015temporalizing} is
\begin{align}
    &\text{"$\Circle^{-}$(SELECT $url$ FROM $autos$ WHERE $price$ $<$ 10000) } \nonumber \\
    &\text{$\wedge$ SELECT $url$ FROM $autos$ WHERE $price$ $>$ 10000".} \nonumber
\end{align}
Query REF QUERY2 can be rewritten as follows:
\begin{itemize}
    \item "[\dots] compared to the last point in time?" indicates the use of the "conjunction" operator from \cite{borgwardt2015temporalizing} with either a "strong previous" or a "weak previous" operator from \cite{borgwardt2015temporalizing} as one argument
    \item Since the query needs to compare the value of one attribute at two different points in time it needs a restriction like "[\dots] WHERE $price1 < price2$" with $price1$ being the average price of the first point in time and $price2$ the average price of the second point in time.
\end{itemize}
Thus, a resulting query might be
\begin{align}
    &\text{"$\Circle^{-}$(SELECT $marke$, AVG($price$) AS $price1$ FROM $autos$)}  \nonumber \\
    &\text{$\wedge$ SELECT $marke$, AVG($price$) AS $price2$ FROM $autos$ WHERE $price1 < price2$".}  \nonumber
\end{align}
Notice, that this kind of restriction is not considered in \cite{borgwardt2015temporalizing} and therefore is also not considered in the implementation of the bounded history encoding from \cite{borgwardt2015temporalizing}.

This also applies to the operators
$\Box \phi$ (always), $\Box^{-} \phi$ (always in the past), $\Diamond \phi$ (eventually), $\Diamond^{-} \phi$
(some time in the past) that were not considered in \cite{borgwardt2015temporalizing} when defining the algorithm.

Lastly, the query REF QUERY3 can be rewritten as follows:
\begin{align}
    &\text{"SELECT $url$ FROM $autos$ WHERE $price$ $>$ 10000} \nonumber \\
    &\text{$\vee \Circle^{-}$(SELECT $url$ FROM $autos$ WHERE $price$ $>$ 10000)} \nonumber \\
    &\text{$\vee \Circle^{-}\Circle^{-}$(SELECT $url$ FROM $autos$ WHERE $price$ $>$ 10000)} \nonumber \\
    &\text{$\vee \Circle^{-}\Circle^{-}\Circle^{-}$(SELECT $url$ FROM $autos$ WHERE $price$ $>$ 10000)} \nonumber \\
    &\text{$\vee \Circle^{-}\Circle^{-}\Circle^{-}\Circle^{-}$(SELECT $url$ FROM $autos$ WHERE $price$ $>$ 10000)} \nonumber \\
    &\text{$\vee \Circle^{-}\Circle^{-}\Circle^{-}\Circle^{-}\Circle^{-}$(SELECT $url$ FROM $autos$ WHERE $price$ $>$ 10000)"} \nonumber
\end{align}
This kind query is highly complex when expressed in the language from \cite{borgwardt2015temporalizing}.
To simplify matters, this query would need a time limit in the form of metric temporal operators.

Consequently, it is necessary to extend the implementation of the bounded history encoding from \cite{borgwardt2015temporalizing} by the missing operators.
Then it can be checked whether cross-referencing of query variables is supported and whether the language from \cite{borgwardt2015temporalizing} can be extended by metric temporal operators.

\subsection{Randomized TQs}
\label{subsec:queries/expressibility/random}
Since only a predefined set of operators is used in the construction of randomized TQs, it is obvious that any randomized
TQ can be expressed in any desired language, i.e. the language from \cite{borgwardt2015temporalizing}, if $\mathsf{Ops}$ is exactly the set of operators available in this language.
In the following $\mathcal{Q}-\text{query } \psi$ is defined as the SQL query "SELECT * FROM autos" which returns every offer without any restriction.


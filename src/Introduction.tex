\chapter*{Introduction}
\addcontentsline{toc}{chapter}{Introduction}
\label{ch:introduction}
This thesis deals with a \textit{bounded history encoding} (BHE) \cite{chomicki1995efficient} and its experimental application on an online automotive marketplace.

The theoretical framework for this thesis is set by \cite{borgwardt2015temporalizing}.
In \cite{borgwardt2015temporalizing}, a temporal version of \textit{ontology-based data access} (OBDA) is considered.
A generic temporal query language is presented.
The reasoning task of the temporal OBDA is reduced to query answering over temporal databases,
the so-called \textit{temporal database monitoring problem}.
Three approaches to solve this problem are presented, including an algorithm that constitutes a BHE.
This work builds upon the latter approach.
The algorithm will be referred to as the BHE, since it is the only one considered.

This thesis ties in with \cite{borgwardt2015temporalizing} and checks the applicability of the algorithm by means of a
concrete application on publicly available data, collected from an online automotive marketplace.
The investigation carried out is structured in four steps.
Chapter \ref{ch:datastream} shows how the publicly accessible data is regularly extracted from an online portal into a database.
In Chapter \ref{ch:queries} an attempt is made, to formulate practical and meaningful queries in the language of \cite{borgwardt2015temporalizing}.
It is to be checked, whether the query language should be extended by additional operators.
Supplementary to practical and meaningful queries, random queries are formulated based on the idea from \cite{schneider2019formally}.
This ensures that the results are valid for the entire language from \cite{borgwardt2015temporalizing}.
The results of the second chapter are used in the third chapter to construct concrete extensions of the already existent query language from \cite{borgwardt2015temporalizing}.
To analyze the algorithm, in Chapter \ref{ch:evaluation} the extended language is applied to the extracted data from the online automotive marketplace.
The goal of this chapter is to evaluate how well the queries can be answered with the help of the BHE \cite{borgwardt2015temporalizing}.
Criteria for this are the size of the encoding, the time to answer one query per time point and the usefulness of the answers.
On the basis of the above-mentioned investigations, the Conclusion summarizes the results in order to answer the research question,
to which extent the BHE is helpful in a practical application.

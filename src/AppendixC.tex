\chapter{Complete Proofs}
\label{ch:appendixC}
\propOperators*
\begin{proof}
    The proof works mainly on the basis of semantics.
    \begin{enumerate}
        \item $\Box \phi_{1} \equiv \phi_{1} \wedge \CIRCLE \Box \phi_{1}$
        \begin{align}
            &\mathfrak{I}, i \models \mathfrak{a}(\Box \phi_{1}) \\
            \Leftrightarrow &\mathfrak{I}, k \models \mathfrak{a}(\phi_{1}) \text{ for all } k, i \leq k \leq n  \label{eq:extension/operatorsnotimplemented/proofProp2/eq1.2App} \\
            \Leftrightarrow &\mathfrak{I}, i \models \mathfrak{a}(\phi_{1}) \text{ and } (i < n \text{ implies } \label{eq:extension/operatorsnotimplemented/proofProp2/eq1.3App}\\
            &\mathfrak{I}, k \models \mathfrak{a}(\phi_{1}) \text{ for all } k, i+1 \leq k \leq n) \nonumber \\
            \Leftrightarrow &\mathfrak{I}, i \models \mathfrak{a}(\phi_{1}) \text{ and } (i < n \text{ implies } \mathfrak{I}, i+1 \models \mathfrak{a}(\Box\phi_{1})) \\
            \Leftrightarrow &\mathfrak{I}, i \models \mathfrak{a}(\phi_{1} \wedge \CIRCLE \Box \phi_{1})
        \end{align}
        \eqref{eq:extension/operatorsnotimplemented/proofProp2/eq1.3App} is equivalent to \eqref{eq:extension/operatorsnotimplemented/proofProp2/eq1.2App} because
        \begin{itemize}
            \item in case $i < n$, the query needs to be satisfied now, at time point $i$, and at all future time points $k$, $i+1 \leq k \leq n$, to be satisfied.
            Since $i < n$ is true, the satisfaction of future time points depends solely on the second part of the  ``implies''-statement; and
            \item in case $i = n$, the query needs to be satisfied now, at time point $i$, to be satisfied.
            There are no future time points $k$, $n+1 \leq k \leq n$.
            Since $i = n$ is true, $\mathfrak{I}, i \models \mathfrak{a}(\phi_{1})$ is equivalent to $\mathfrak{I}, k \models \mathfrak{a}(\phi_{1})$ for all $k$, $n \leq k \leq n$,
            and $i < n$ is not true, thus the  ``implies''-statement does not affect satisfaction.
        \end{itemize}

        \item $\Box^{-} \phi_{1} \equiv \phi_{1} \wedge \CIRCLE^{-} \Box^{-} \phi_{1}$
        \begin{align}
            &\mathfrak{I}, i \models \mathfrak{a}(\Box^{-} \phi_{1}) \\
            \Leftrightarrow &\mathfrak{I}, k \models \mathfrak{a}(\phi_{1}) \text{ for all } k, 0 \leq k \leq i \label{eq:extension/operatorsnotimplemented/proofProp2/eq2.2App}\\
            \Leftrightarrow &\mathfrak{I}, i \models \mathfrak{a}(\phi_{1}) \text{ and } (i > 0 \text{ implies } \label{eq:extension/operatorsnotimplemented/proofProp2/eq2.3App}\\
            &\mathfrak{I}, k \models \mathfrak{a}(\phi_{1}) \text{ for all } k, 0 \leq k \leq i-1) \nonumber \\
            \Leftrightarrow &\mathfrak{I}, i \models \mathfrak{a}(\phi_{1}) \text{ and } (i > 0 \text{ implies } \mathfrak{I}, i-1 \models \mathfrak{a}(\Box^{-}\phi_{1})) \\
            \Leftrightarrow &\mathfrak{I}, i \models \mathfrak{a}(\phi_{1} \wedge \CIRCLE^{-} \Box^{-} \phi_{1})
        \end{align}
        \eqref{eq:extension/operatorsnotimplemented/proofProp2/eq2.3App} is equivalent to \eqref{eq:extension/operatorsnotimplemented/proofProp2/eq2.2App} because
        \begin{itemize}
            \item in case $i > 0$, the query needs to be satisfied now, at time point $i$, and at all past time points $k$, $0 \leq k \leq i-1$, to be satisfied.
            Since $i > 0$ is true, the satisfaction of past time points depends solely on the second part of the  ``implies''-statement; and
            \item in case $i = 0$, the query needs to be satisfied now, at time point $i$, to be satisfied.
            There are no past time points $k$, $0 \leq k \leq 0-1$.
            Since $i = 0$ is true, $\mathfrak{I}, i \models \mathfrak{a}(\phi_{1})$ is equivalent to $\mathfrak{I}, k \models \mathfrak{a}(\phi_{1})$ for all $k$, $0 \leq k \leq 0$,
            and $i > 0$ is not true, thus the  ``implies''-statement does not affect satisfaction.
        \end{itemize}

        \item $\Diamond \phi_{1} \equiv \phi_{1} \vee \Circle \Diamond \phi_{1}$
        \begin{align}
            &\mathfrak{I}, i \models \mathfrak{a}(\Diamond \phi_{1}) \\
            \Leftrightarrow &\mathfrak{I}, k \models \mathfrak{a}(\phi_{1}) \text{ for some } k, i \leq k \leq n \label{eq:extension/operatorsnotimplemented/proofProp2/eq3.2App}\\
            \Leftrightarrow &\mathfrak{I}, i \models \mathfrak{a}(\phi_{1}) \text{ or } (i < n \text{ and } \label{eq:extension/operatorsnotimplemented/proofProp2/eq3.3App}\\
            &\mathfrak{I}, k \models \mathfrak{a}(\phi_{1}) \text{ for some } k, i+1 \leq k \leq n) \nonumber \\
            \Leftrightarrow &\mathfrak{I}, i \models \mathfrak{a}(\phi_{1}) \text{ or } (i < n \text{ and } \mathfrak{I}, i+1 \models \mathfrak{a}(\Diamond \phi_{1})) \\
            \Leftrightarrow &\mathfrak{I}, i \models \mathfrak{a}(\phi_{1} \vee \Circle \Diamond \phi_{1})
        \end{align}
        \eqref{eq:extension/operatorsnotimplemented/proofProp2/eq3.3App} is equivalent to \eqref{eq:extension/operatorsnotimplemented/proofProp2/eq3.2App} because
        \begin{itemize}
            \item in case $i < n$, the query needs to be satisfied now, at time point $i$, or at any future time point $k$, $i+1 \leq k \leq n$, to be satisfied.
            Since $i < n$ is true, the satisfaction of future time points depends solely on the second part of the  ``and''-statement; and
            \item in case $i = n$, the query needs to be satisfied now, at time point $i$, to be satisfied.
            There are no future time points $k$, $n+1 \leq k \leq n$.
            Since $i = n$ is true, $\mathfrak{I}, i \models \mathfrak{a}(\phi_{1})$ is equivalent to $\mathfrak{I}, k \models \mathfrak{a}(\phi_{1})$ for some $k$, $n \leq k \leq n$,
            and $i > 0$ is not true, thus the  ``and''-statement does not affect satisfaction.
        \end{itemize}

        \item $\Diamond^{-} \phi_{1} \equiv \phi_{1} \vee \Circle^{-} \Diamond^{-} \phi_{1}$
        \begin{align}
            &\mathfrak{I}, i \models \mathfrak{a}(\Diamond^{-} \phi_{1}) \\
            \Leftrightarrow &\mathfrak{I}, k \models \mathfrak{a}(\phi_{1}) \text{ for some } k, 0 \leq k \leq i \label{eq:extension/operatorsnotimplemented/proofProp2/eq4.2App}\\
            \Leftrightarrow &\mathfrak{I}, i \models \mathfrak{a}(\phi_{1}) \text{ or } (i > 0 \text{ and } \label{eq:extension/operatorsnotimplemented/proofProp2/eq4.3App}\\
            &\mathfrak{I}, k \models \mathfrak{a}(\phi_{1}) \text{ for some } k, 0 \leq k \leq i-1) \nonumber \\
            \Leftrightarrow &\mathfrak{I}, i \models \mathfrak{a}(\phi_{1}) \text{ or } (i > 0 \text{ and } \mathfrak{I}, i-1 \models \mathfrak{a}(\Diamond^{-} \phi_{1})) \\
            \Leftrightarrow &\mathfrak{I}, i \models \mathfrak{a}(\phi_{1} \vee \Circle^{-} \Diamond^{-} \phi_{1})
        \end{align}
        \eqref{eq:extension/operatorsnotimplemented/proofProp2/eq4.3App} is equivalent to \eqref{eq:extension/operatorsnotimplemented/proofProp2/eq4.2App} because
        \begin{itemize}
            \item in case $i > 0$, the query needs to be satisfied now, at time point $i$, or at any past time point $k$, $0 \leq k \leq i-1$, to be satisfied.
            Since $i > 0$ is true, the satisfaction of past time points depends solely on the second part of the  ``and''-statement; and
            \item in case $i = 0$, the query needs to be satisfied now, at time point $i$, to be satisfied.
            There are no past time points $k$, $0 \leq k \leq 0-1$.
            Since $i = 0$ is true, $\mathfrak{I}, i \models \mathfrak{a}(\phi_{1})$ is equivalent to $\mathfrak{I}, k \models \mathfrak{a}(\phi_{1})$ for some $k$, $0 \leq k \leq 0$,
            and $i > 0$ is not true, thus the  ``and''-statement does not affect satisfaction.
        \end{itemize}
    \end{enumerate}
\end{proof}

\lemmaOperators*
\begin{proof}
    It is shown by induction on the structure of the subqueries $\psi \in \mathsf{Sub}(\phi)$ that $\mathsf{eval}^{n}(\Phi^{0}_{i}(\psi))$
    is equal to $\mathsf{Ans}(\psi, \mathfrak{I}^{(n)},i)$ for all $n \geq i$. \\
    If $\psi = \Box^{-}\psi_{1}$, then
    \begin{equation}
        \notag
        \label{eq:extension/operatorsnotimplemented/proofLemma6/eq1App}
        \begin{split}
            \mathsf{eval}^{n}(\Phi^{0}_{i}(\psi)) &= \mathsf{eval}^{n}(\Phi^{0}_{i}(\psi_{1})) \cap \mathsf{eval}^{n}(\Phi_{i-1}(\psi)) \\
            &= \mathsf{Ans}(\psi_{1}, \mathfrak{I}^{(n)},i) \cap \mathsf{Ans}(\psi, \mathfrak{I}^{(n)},i-1) \\
            &= \mathsf{Ans}(\psi, \mathfrak{I}^{(n)},i)
        \end{split}
    \end{equation}
    If $\psi = \Diamond^{-}\psi_{1}$, then
    \begin{equation}
        \notag
        \label{eq:extension/operatorsnotimplemented/proofLemma6/eq2App}
        \begin{split}
            \mathsf{eval}^{n}(\Phi^{0}_{i}(\psi)) &= \mathsf{eval}^{n}(\Phi^{0}_{i}(\psi_{1})) \cup \mathsf{eval}^{n}(\Phi_{i-1}(\psi)) \\
            &= \mathsf{Ans}(\psi_{1}, \mathfrak{I}^{(n)},i) \cup \mathsf{Ans}(\psi, \mathfrak{I}^{(n)},i-1) \\
            &= \mathsf{Ans}(\psi, \mathfrak{I}^{(n)},i)
        \end{split}
    \end{equation}
    If $\psi = \Box\psi_{1}$, then
    \begin{equation}
        \notag
        \label{eq:extension/operatorsnotimplemented/proofLemma6/eq3App}
        \begin{split}
            \mathsf{eval}^{n}(\Phi^{0}_{i}(\psi)) &= \mathsf{eval}^{n}(\Phi^{0}_{i}(\psi_{1})) \cap \mathsf{eval}^{n}(x^{\psi}_{i}) \\
            &= \mathsf{Ans}(\psi_{1}, \mathfrak{I}^{(n)},i) \cap \left\{\begin{array}{lr}
                                                                            \mathsf{Ans}(\psi, \mathfrak{I}^{(n)},i+1)  & \text{if } n > i\\
                                                                            \Delta^{\mathsf{N}_\mathsf{V}}          & \text{if } n = i\\
            \end{array}\right\} \\
            &= \mathsf{Ans}(\psi, \mathfrak{I}^{(n)},i)
        \end{split}
    \end{equation}
    If $\psi = \Diamond\psi_{1}$, then
    \begin{equation}
        \notag
        \label{eq:extension/operatorsnotimplemented/proofLemma6/eq4App}
        \begin{split}
            \mathsf{eval}^{n}(\Phi^{0}_{i}(\psi)) &= \mathsf{eval}^{n}(\Phi^{0}_{i}(\psi_{1})) \cup \mathsf{eval}^{n}(x^{\psi}_{i}) \\
            &= \mathsf{Ans}(\psi_{1}, \mathfrak{I}^{(n)},i) \cup \left\{\begin{array}{lr}
                                                                            \mathsf{Ans}(\psi, \mathfrak{I}^{(n)},i+1)  & \text{if } n > i\\
                                                                            \emptyset               & \text{if } n = i\\
            \end{array}\right\} \\
            &= \mathsf{Ans}(\psi, \mathfrak{I}^{(n)},i)
        \end{split}
    \end{equation}
\end{proof}

\propMTOsPzero*
\begin{proof}
    The proof works mainly on the basis of semantics.
    \begin{enumerate}
        \item $\Circle_{0} \phi_{1} \equiv \phi_{1}$
        \begin{align}
            &\mathfrak{I},i \models \mathfrak{a}(\Circle_{0} \phi_{1}) \\
            \Leftrightarrow &i+0 \leq n\text{ and }\mathfrak{I},i+0 \models \mathfrak{a}(\phi_{1}) \\
            \Leftrightarrow &\mathfrak{I}, i \models \mathfrak{a}(\phi_{1})
        \end{align}

        \item $\CIRCLE_{0} \phi_{1} \equiv \phi_{1}$
        \begin{align}
            &\mathfrak{I},i \models \mathfrak{a}(\CIRCLE_{0} \phi_{1}) \\
            \Leftrightarrow &i+0 \leq n\text{ implies }\mathfrak{I},i+0 \models \mathfrak{a}(\phi_{1}) \\
            \Leftrightarrow &\mathfrak{I}, i \models \mathfrak{a}(\phi_{1})
        \end{align}

        \item $\Circle^{-}_{0} \phi_{1} \equiv \phi_{1}$
        \begin{align}
            &\mathfrak{I},i \models \mathfrak{a}(\Circle^{-}_{0} \phi_{1}) \\
            \Leftrightarrow &i-0 \geq 0\text{ and }\mathfrak{I},i-0 \models \mathfrak{a}(\phi_{1}) \\
            \Leftrightarrow &\mathfrak{I}, i \models \mathfrak{a}(\phi_{1})
        \end{align}

        \item $\CIRCLE^{-}_{0} \phi_{1} \equiv \phi_{1}$
        \begin{align}
            &\mathfrak{I},i \models \mathfrak{a}(\CIRCLE^{-}_{0} \phi_{1}) \\
            \Leftrightarrow &i-0 \geq 0\text{ implies }\mathfrak{I},i-0 \models \mathfrak{a}(\phi_{1}) \\
            \Leftrightarrow &\mathfrak{I}, i \models \mathfrak{a}(\phi_{1})
        \end{align}

        \item $\Box_{0} \phi_{1} \equiv \phi_{1}$
        \begin{align}
            &\mathfrak{I}, i \models \mathfrak{a}(\Box_{0} \phi_{1}) \\
            \Leftrightarrow &\mathfrak{I}, k \models \mathfrak{a}(\phi_{1}) \text{ for all } k, i \leq k \leq \mathsf{min}(i+0,n) \\
            \Leftrightarrow &\mathfrak{I}, i \models \mathfrak{a}(\phi_{1})
        \end{align}

        \item $\Box^{-}_{0} \phi_{1} \equiv \phi_{1}$
        \begin{align}
            &\mathfrak{I}, i \models \mathfrak{a}(\Box^{-}_{0} \phi_{1}) \\
            \Leftrightarrow &\mathfrak{I}, k \models \mathfrak{a}(\phi_{1}) \text{ for all } k, \mathsf{max}(i-0,0) \leq k \leq i \\
            \Leftrightarrow &\mathfrak{I}, i \models \mathfrak{a}(\phi_{1})
        \end{align}

        \item $\Diamond_{0} \phi_{1} \equiv \phi_{1}$
        \begin{align}
            &\mathfrak{I}, i \models \mathfrak{a}(\Diamond_{0} \phi_{1}) \\
            \Leftrightarrow &\mathfrak{I}, k \models \mathfrak{a}(\phi_{1}) \text{ for some } k, i \leq k \leq \mathsf{min}(i+0,n) \\
            \Leftrightarrow &\mathfrak{I}, i \models \mathfrak{a}(\phi_{1})
        \end{align}

        \item $\Diamond^{-}_{0} \phi_{1} \equiv \phi_{1}$
        \begin{align}
            &\mathfrak{I}, i \models \mathfrak{a}(\Diamond^{-}_{0} \phi_{1}) \\
            \Leftrightarrow &\mathfrak{I}, k \models \mathfrak{a}(\phi_{1}) \text{ for some } k, \mathsf{max}(i-0,0) \leq k \leq i \\
            \Leftrightarrow &\mathfrak{I}, i \models \mathfrak{a}(\phi_{1})
        \end{align}

        \item $\phi_{1} \mathsf{U}_{0} \phi_{2} \equiv \phi_{2}$
        \begin{align}
            &\mathfrak{I}, i \models \mathfrak{a}(\phi_{1} \mathsf{U}_{0} \phi_{2}) \\
            \Leftrightarrow &\text{there is } k, i \leq k \leq \mathsf{min}(i+0,n)\text{, with } \mathfrak{I}, k \models \mathfrak{a}_{\phi_{2}}(\phi_{2}) \text{ and } \\
            &\mathfrak{I}, j \models \mathfrak{a}_{\phi_{1}}(\phi_{1})\text{ for all }j, i \leq j < k \nonumber \\
            \Leftrightarrow &\mathfrak{I}, i \models \mathfrak{a}_{\phi_{2}}(\phi_{2})
        \end{align}

        \item $\phi_{1} \mathsf{S}_{0} \phi_{2} \equiv \phi_{2}$
        \begin{align}
            &\mathfrak{I}, i \models \mathfrak{a}(\phi_{1} \mathsf{S}_{0} \phi_{2}) \\
            \Leftrightarrow &\text{there is } k, \mathsf{max}(i-0,0) \leq k \leq i\text{, with } \mathfrak{I}, k \models \mathfrak{a}_{\phi_{2}}(\phi_{2}) \text{ and } \\
            &\mathfrak{I}, j \models \mathfrak{a}_{\phi_{1}}(\phi_{1})\text{ for all }j, k < j \leq i \nonumber \\
            \Leftrightarrow &\mathfrak{I}, i \models \mathfrak{a}_{\phi_{2}}(\phi_{2})
        \end{align}
    \end{enumerate}
\end{proof}

\propMTOs*
\begin{proof}
    The proof works mainly on the basis of semantics.
    \begin{enumerate}
        \item $\Circle_{p} \phi_{1} \equiv \Circle \Circle_{p-1} \phi_{1}$
        \begin{align}
            &\mathfrak{I}, i \models \mathfrak{a}(\Circle_{p} \phi_{1}) \\
            \Leftrightarrow &i + p \leq n \text{ and }\mathfrak{I}, i+p \models \mathfrak{a}(\phi_{1}) \label{eq:extension/metrictemporaloperators/proofProp3/eq1.2App}\\
            \Leftrightarrow & i < n\text{ and }(i+1)+(p-1) \leq n \text{ and }\mathfrak{I}, (i+1)+(p-1) \models \mathfrak{a}(\phi_{1}) \label{eq:extension/metrictemporaloperators/proofProp3/eq1.3App}\\
            \Leftrightarrow & i < n\text{ and }\mathfrak{I}, i+1 \models \mathfrak{a}(\Circle_{p-1} \phi_{1}) \\
            \Leftrightarrow &\mathfrak{I}, i \models \mathfrak{a}(\Circle \Circle_{p-1} \phi_{1})
        \end{align}
        \item $\CIRCLE_{p} \phi_{1}$ is equivalent to $\CIRCLE \CIRCLE_{p-1} \phi_{1}$
        \begin{align}
            &\mathfrak{I}, i \models \mathfrak{a}(\CIRCLE_{p} \phi_{1}) \\
            \Leftrightarrow &i + p \leq n \text{ implies }\mathfrak{I}, i+p \models \mathfrak{a}(\phi_{1}) \label{eq:extension/metrictemporaloperators/proofProp3/eq2.2App}\\
            \Leftrightarrow & i < n\text{ implies }(i+1)+(p-1) \leq n \text{ implies }\mathfrak{I}, (i+1)+(p-1) \models \mathfrak{a}(\phi_{1}) \label{eq:extension/metrictemporaloperators/proofProp3/eq2.3App}\\
            \Leftrightarrow & i < n\text{ implies }\mathfrak{I}, i+1 \models \mathfrak{a}(\CIRCLE_{p-1} \phi_{1}) \\
            \Leftrightarrow &\mathfrak{I}, i \models \mathfrak{a}(\CIRCLE \CIRCLE_{p-1} \phi_{1})
        \end{align}
        \item $\Circle^{-}_{p} \phi_{1}$ is equivalent to $\Circle^{-} \Circle^{-}_{p-1} \phi_{1}$
        \begin{align}
            &\mathfrak{I}, i \models \mathfrak{a}(\Circle^{-}_{p} \phi_{1}) \\
            \Leftrightarrow &i - p \geq 0 \text{ and }\mathfrak{I}, i-p \models \mathfrak{a}(\phi_{1}) \label{eq:extension/metrictemporaloperators/proofProp3/eq3.2App}\\
            \Leftrightarrow & i > 0\text{ and }(i-1)-(p-1) \geq 0 \text{ and }\mathfrak{I}, (i-1)-(p-1) \models \mathfrak{a}(\phi_{1}) \label{eq:extension/metrictemporaloperators/proofProp3/eq3.3App}\\
            \Leftrightarrow & i > 0\text{ and }\mathfrak{I}, i-1 \models \mathfrak{a}(\Circle^{-}_{p-1} \phi_{1}) \\
            \Leftrightarrow &\mathfrak{I}, i \models \mathfrak{a}(\Circle^{-} \Circle^{-}_{p-1} \phi_{1})
        \end{align}
        \item $\CIRCLE^{-}_{p} \phi_{1}$ is equivalent to $\CIRCLE^{-} \CIRCLE^{-}_{p-1} \phi_{1}$
        \begin{align}
            &\mathfrak{I}, i \models \mathfrak{a}(\CIRCLE^{-}_{p} \phi_{1}) \\
            \Leftrightarrow &i - p \geq 0 \text{ implies }\mathfrak{I}, i-p \models \mathfrak{a}(\phi_{1}) \label{eq:extension/metrictemporaloperators/proofProp3/eq4.2App}\\
            \Leftrightarrow & i > 0\text{ implies }(i-1)-(p-1) \geq 0 \text{ implies }\mathfrak{I}, (i-1)-(p-1) \models \mathfrak{a}(\phi_{1}) \label{eq:extension/metrictemporaloperators/proofProp3/eq4.3App}\\
            \Leftrightarrow & i > 0\text{ implies }\mathfrak{I}, i-1 \models \mathfrak{a}(\CIRCLE^{-}_{p-1} \phi_{1}) \\
            \Leftrightarrow &\mathfrak{I}, i \models \mathfrak{a}(\CIRCLE^{-} \CIRCLE^{-}_{p-1} \phi_{1})
        \end{align}
        \item $\Box_{p} \phi_{1} \equiv \phi_{1} \wedge \CIRCLE \Box_{p-1} \phi_{1}$
        \begin{align}
            &\mathfrak{I}, i \models \mathfrak{a}(\Box_{p} \phi_{1}) \\
            \Leftrightarrow &\mathfrak{I}, k \models \mathfrak{a}(\phi_{1}) \text{ for all } k, i \leq k \leq \mathsf{min}(i+p,n) \label{eq:extension/metrictemporaloperators/proofProp3/eq5.2App}\\
            \Leftrightarrow &\mathfrak{I}, i \models \mathfrak{a}(\phi_{1}) \text{ and } (i < n \text{ implies } \label{eq:extension/metrictemporaloperators/proofProp3/eq5.3App}\\
            &\mathfrak{I}, k \models \mathfrak{a}(\phi_{1}) \text{ for all } k, i+1 \leq k \leq \mathsf{min}((i+1)+(p-1),n)) \nonumber \\
            \Leftrightarrow &\mathfrak{I}, i \models \mathfrak{a}(\phi_{1}) \text{ and } (i < n \text{ implies } \mathfrak{I}, i+1 \models \mathfrak{a}(\Box_{p-1} \phi_{1})) \\
            \Leftrightarrow &\mathfrak{I}, i \models \mathfrak{a}(\phi_{1} \wedge \CIRCLE \Box_{p-1} \phi_{1})
        \end{align}
        \eqref{eq:extension/metrictemporaloperators/proofProp3/eq5.3App} is equivalent to \eqref{eq:extension/metrictemporaloperators/proofProp3/eq5.2App} because
        \begin{itemize}
            \item in case $i < n$, the query needs to be satisfied now, at time point $i$, and at future time points $k$, $i+1 \leq k \leq \mathsf{min}((i+1)+(p-1),n)$, to be satisfied.
            Since $i < n$ is true, the satisfaction of future time points depends solely on the second part of the ``implies''-statement; and
            \item in case $i = n$, the query needs to be satisfied now, at time point $i$, to be satisfied.
            There are no future time points $k$, $n+1 \leq k \leq \mathsf{min}((i+1)+(p-1),n)$.
            Since $i = n$ is true, $\mathfrak{I}, i \models \mathfrak{a}(\phi_{1})$ is equivalent to $\mathfrak{I}, k \models \mathfrak{a}(\phi_{1})$ for all $k$, $n \leq k \leq \mathsf{min}(i+p,n)$,
            and $i < n$ is not true, thus the ``implies''-statement does not affect satisfaction.
        \end{itemize}

        \item $\Box^{-}_{p} \phi_{1} \equiv \phi_{1} \wedge \CIRCLE^{-} \Box^{-}_{p} \phi_{1}$
        \begin{align}
            &\mathfrak{I}, i \models \mathfrak{a}(\Box^{-}_{p} \phi_{1}) \\
            \Leftrightarrow &\mathfrak{I}, k \models \mathfrak{a}(\phi_{1}) \text{ for all } k, \mathsf{max}(i-p,0) \leq k \leq i \label{eq:extension/metrictemporaloperators/proofProp3/eq6.2App}\\
            \Leftrightarrow &\mathfrak{I}, i \models \mathfrak{a}(\phi_{1}) \text{ and } (i > 0 \text{ implies } \label{eq:extension/metrictemporaloperators/proofProp3/eq6.3App}\\
            &\mathfrak{I}, k \models \mathfrak{a}(\phi_{1}) \text{ for all } k, \mathsf{max}((i-1)-(p-1),0) \leq k \leq i-1) \nonumber \\
            \Leftrightarrow &\mathfrak{I}, i \models \mathfrak{a}(\phi_{1}) \text{ and } (i > 0 \text{ implies } \mathfrak{I}, i-1 \models \mathfrak{a}(\Box^{-}_{p-1} \phi_{1})) \\
            \Leftrightarrow &\mathfrak{I}, i \models \mathfrak{a}(\phi_{1} \wedge \CIRCLE^{-} \Box^{-}_{p-1} \phi_{1})
        \end{align}
        \eqref{eq:extension/metrictemporaloperators/proofProp3/eq6.3App} is equivalent to \eqref{eq:extension/metrictemporaloperators/proofProp3/eq6.2App} because
        \begin{itemize}
            \item in case $i > 0$, the query needs to be satisfied now, at time point $i$, and at past time points $k$, $\mathsf{max}((i-1)-(p-1),0) \leq k \leq i-1$, to be satisfied.
            Since $i > 0$ is true, the satisfaction of past time points depends solely on the second part of the ``implies''-statement; and
            \item in case $i = 0$, the query needs to be satisfied now, at time point $i$, to be satisfied.
            There are no past time points $k$, $\mathsf{max}((i-1)-(p-1),0) \leq k \leq 0-1$.
            Since $i = 0$ is true, $\mathfrak{I}, i \models \mathfrak{a}(\phi_{1})$ is equivalent to $\mathfrak{I}, k \models \mathfrak{a}(\phi_{1})$ for all $k$, $\mathsf{max}(i-p,0) \leq k \leq 0$,
            and $i > 0$ is not true, thus the ``implies''-statement does not affect satisfaction.
        \end{itemize}

        \item $\Diamond_{p} \phi_{1} \equiv \phi_{1} \vee \Circle \Diamond_{p} \phi_{1}$
        \begin{align}
            &\mathfrak{I}, i \models \mathfrak{a}(\Diamond_{p} \phi_{1}) \\
            \Leftrightarrow &\mathfrak{I}, k \models \mathfrak{a}(\phi_{1}) \text{ for some } k, i \leq k \leq \mathsf{min}(i+p,n) \label{eq:extension/metrictemporaloperators/proofProp3/eq7.2App}\\
            \Leftrightarrow &\mathfrak{I}, i \models \mathfrak{a}(\phi_{1}) \text{ or } (i < n \text{ and } \label{eq:extension/metrictemporaloperators/proofProp3/eq7.3App}\\
            &\mathfrak{I}, k \models \mathfrak{a}(\phi_{1}) \text{ for some } k, i+1 \leq k \leq \mathsf{min}((i+1)+(p-1),n)) \nonumber \\
            \Leftrightarrow &\mathfrak{I}, i \models \mathfrak{a}(\phi_{1}) \text{ or } (i < n \text{ and } \mathfrak{I}, i+1 \models \mathfrak{a}(\Diamond_{p-1} \phi_{1})) \\
            \Leftrightarrow &\mathfrak{I}, i \models \mathfrak{a}(\phi_{1} \vee \Circle \Diamond_{p-1} \phi_{1})
        \end{align}
        \eqref{eq:extension/metrictemporaloperators/proofProp3/eq7.3App} is equivalent to \eqref{eq:extension/metrictemporaloperators/proofProp3/eq7.2App} because
        \begin{itemize}
            \item in case $i < n$, the query needs to be satisfied now, at time point $i$, or at any of the future time points $k$, $i+1 \leq k \leq \mathsf{min}((i+1)+(p-1),n)$, to be satisfied.
            Since $i < n$ is true, the satisfaction of future time points depends solely on the second part of the ``and''-statement; and
            \item in case $i = n$, the query needs to be satisfied now, at time point $i$, to be satisfied.
            There are no future time points $k$, $n+1 \leq k \leq \mathsf{min}((i+1)+(p-1),n)$.
            Since $i = n$ is true, $\mathfrak{I}, i \models \mathfrak{a}(\phi_{1})$ is equivalent to $\mathfrak{I}, k \models \mathfrak{a}(\phi_{1})$ for some $k$, $n \leq k \leq \mathsf{min}(i+p,n)$,
            and $i < n$ is not true, thus the ``and''-statement does not affect satisfaction.
        \end{itemize}

        \item $\Diamond^{-}_{p} \phi_{1} \equiv \phi_{1} \vee \Circle^{-} \Diamond^{-}_{p} \phi_{1}$
        \begin{align}
            &\mathfrak{I}, i \models \mathfrak{a}(\Diamond^{-}_{p} \phi_{1}) \\
            \Leftrightarrow &\mathfrak{I}, k \models \mathfrak{a}(\phi_{1}) \text{ for some } k, \mathsf{max}(i-p,0) \leq k \leq i \label{eq:extension/metrictemporaloperators/proofProp3/eq8.2App}\\
            \Leftrightarrow &\mathfrak{I}, i \models \mathfrak{a}(\phi_{1}) \text{ or } (i > 0 \text{ and } \label{eq:extension/metrictemporaloperators/proofProp3/eq8.3App}\\
            &\mathfrak{I}, k \models \mathfrak{a}(\phi_{1}) \text{ for some } k, \mathsf{max}((i-1)-(p-1),0) \leq k \leq i-1) \nonumber \\
            \Leftrightarrow &\mathfrak{I}, i \models \mathfrak{a}(\phi_{1}) \text{ or } (i > 0 \text{ and } \mathfrak{I}, i-1 \models \mathfrak{a}(\Diamond^{-}_{p-1} \phi_{1})) \\
            \Leftrightarrow &\mathfrak{I}, i \models \mathfrak{a}(\phi_{1} \vee \Circle^{-} \Diamond^{-}_{p-1} \phi_{1})
        \end{align}
        \eqref{eq:extension/metrictemporaloperators/proofProp3/eq8.3App} is equivalent to \eqref{eq:extension/metrictemporaloperators/proofProp3/eq8.2App} because
        \begin{itemize}
            \item in case $i > 0$, the query needs to be satisfied now, at time point $i$, or at any of the past time points $k$, $\mathsf{max}((i-1)-(p-1),0) \leq k \leq i-1$, to be satisfied.
            Since $i > 0$ is true, the satisfaction of past time points depends solely on the second part of the ``and''-statement; and
            \item in case $i = 0$, the query needs to be satisfied now, at time point $i$, to be satisfied.
            There are no past time points $k$, $\mathsf{max}((i-1)-(p-1),0) \leq k \leq 0-1$.
            Since $i = 0$ is true, $\mathfrak{I}, i \models \mathfrak{a}(\phi_{1})$ is equivalent to $\mathfrak{I}, k \models \mathfrak{a}(\phi_{1})$ for some $k$, $\mathsf{max}(i-p,0) \leq k \leq 0$,
            and $i > 0$ is not true, thus the ``and''-statement does not affect satisfaction.
        \end{itemize}

        \item $\phi_{1} \mathsf{U}_{p} \phi_{2} \equiv \phi_{2} \vee (\phi_{1} \wedge \Circle (\phi_{1} \mathsf{U}_{p-1} \phi_{2}))$
        \begin{align}
            &\mathfrak{I}, i \models \mathfrak{a}(\phi_{1} \mathsf{U}_{p} \phi_{2}) \\
            \Leftrightarrow &\text{there is } k, i \leq k \leq \mathsf{min}(i+p,n)\text{, with } \mathfrak{I}, k \models \mathfrak{a}_{\phi_{2}}(\phi_{2}) \text{ and } \label{eq:extension/metrictemporaloperators/proofProp3/eq9.2App}\\
            &\mathfrak{I}, j \models \mathfrak{a}_{\phi_{1}}(\phi_{1})\text{ for all }j, i \leq j < k \nonumber \\
            \Leftrightarrow &\mathfrak{I}, i \models \mathfrak{a}_{\phi_{2}}(\phi_{2})\text{ or }(\mathfrak{I}, i \models \mathfrak{a}_{\phi_{1}}(\phi_{1})\text{ and }(i < n\text{ and } \label{eq:extension/metrictemporaloperators/proofProp3/eq9.3App}\\
            &\text{there is } k, i+1 \leq k \leq \mathsf{min}((i+1)+(p-1),n)\text{, with }  \nonumber \\
            &\mathfrak{I}, k \models \mathfrak{a}_{\phi_{2}}(\phi_{2}) \text{ and }\mathfrak{I}, j \models \mathfrak{a}_{\phi_{1}}(\phi_{1})\text{ for all }j, i+1 \leq j < k)) \nonumber \\
            \Leftrightarrow &\mathfrak{I}, i \models \mathfrak{a}_{\phi_{2}}(\phi_{2})\text{ or }(\mathfrak{I}, i \models \mathfrak{a}_{\phi_{1}}(\phi_{1})\text{ and }(i < n\text{ and } \\
            &\mathfrak{I}, i+1 \models \mathfrak{a}(\phi_{1} \mathsf{U}_{p-1} \phi_{2}))) \nonumber \\
            \Leftrightarrow &\mathfrak{I}, i \models \mathfrak{a}(\phi_{2} \vee (\phi_{1} \wedge \Circle (\phi_{1} \mathsf{U}_{p-1} \phi_{2})))
        \end{align}
        \eqref{eq:extension/metrictemporaloperators/proofProp3/eq9.3App} is equivalent to \eqref{eq:extension/metrictemporaloperators/proofProp3/eq9.2App} because
        \begin{itemize}
            \item in case $i < n$, either $\phi_{2}$ needs to be satisfied now, at time point $i$, or $\phi_{1}$ needs to be satisfied now, at time point $i$, and there needs to be a future time point $k$, $i+1 \leq k \leq \mathsf{min}((i+1)+(p-1),n)$,
            where $\phi_{2}$ is satisfied and $\phi_{1}$ is satisfied for all time points $j$, $i+1 \leq j < k$, for the query to be satisfied.
            \item in case $i = n$, $\phi_{2}$ needs to be satisfied now, at time point $i$, for the query to be satisfied.
            There are no future time points $k$, $n+1 \leq k \leq \mathsf{min}((i+1)+(p-1),n)$.
            Since $i = n$ is true, $\mathfrak{I}, i \models \mathfrak{a}_{\phi_{2}}(\phi_{2})$ is equivalent to there is $k$, $n \leq k \leq \mathsf{min}(i+p,n)$, with $\mathfrak{I}, k \models \mathfrak{a}_{\phi_{2}}(\phi_{2})$
            and $\mathfrak{I}, j \models \mathfrak{a}_{\phi_{1}}(\phi_{1})$ for all $j, n \leq j < k$, and $i < n$ is not true, thus the ``or''-statement does not affect satisfaction.
        \end{itemize}

        \item $\phi_{1} \mathsf{S}_{p} \phi_{2} \equiv \phi_{2} \vee (\phi_{1} \wedge \Circle^{-} (\phi_{1} \mathsf{S}_{p-1} \phi_{2}))$
        \begin{align}
            &\mathfrak{I}, i \models \mathfrak{a}(\phi_{1} \mathsf{S}_{p} \phi_{2}) \\
            \Leftrightarrow &\text{there is } k, \mathsf{max}(i-p,0) \leq k \leq i\text{, with } \mathfrak{I}, k \models \mathfrak{a}_{\phi_{2}}(\phi_{2}) \text{ and } \label{eq:extension/metrictemporaloperators/proofProp3/eq10.2App}\\
            &\mathfrak{I}, j \models \mathfrak{a}_{\phi_{1}}(\phi_{1})\text{ for all }j, k < j \leq i \nonumber \\
            \Leftrightarrow &\mathfrak{I}, i \models \mathfrak{a}_{\phi_{2}}(\phi_{2})\text{ or }(\mathfrak{I}, i \models \mathfrak{a}_{\phi_{1}}(\phi_{1})\text{ and }(i > 0\text{ and } \label{eq:extension/metrictemporaloperators/proofProp3/eq10.3App}\\
            &\text{there is } k, \mathsf{max}((i-1)-(p-1),0) \leq k \leq i-1\text{, with }  \nonumber \\
            &\mathfrak{I}, k \models \mathfrak{a}_{\phi_{2}}(\phi_{2}) \text{ and }\mathfrak{I}, j \models \mathfrak{a}_{\phi_{1}}(\phi_{1})\text{ for all }j, k < j \leq i-1)) \nonumber \\
            \Leftrightarrow &\mathfrak{I}, i \models \mathfrak{a}_{\phi_{2}}(\phi_{2})\text{ or }(\mathfrak{I}, i \models \mathfrak{a}_{\phi_{1}}(\phi_{1})\text{ and }(i > 0\text{ and } \\
            &\mathfrak{I}, i-1 \models \mathfrak{a}(\phi_{1} \mathsf{S}_{p-1} \phi_{2}))) \nonumber \\
            \Leftrightarrow &\mathfrak{I}, i \models \mathfrak{a}(\phi_{2} \vee (\phi_{1} \wedge \Circle^{-} (\phi_{1} \mathsf{S}_{p-1} \phi_{2})))
        \end{align}
        \eqref{eq:extension/metrictemporaloperators/proofProp3/eq10.3App} is equivalent to \eqref{eq:extension/metrictemporaloperators/proofProp3/eq10.2App} because
        \begin{itemize}
            \item in case $i > 0$, either $\phi_{2}$ needs to be satisfied now, at time point $i$, or $\phi_{1}$ needs to be satisfied now, at time point $i$, and there needs to be a past time point $k$, $\mathsf{max}((i-1)-(p-1),0) \leq k \leq i-1$,
            where $\phi_{2}$ is satisfied and $\phi_{1}$ is satisfied for all time points $j$, $k < j \leq i-1$, for the query to be satisfied.
            \item in case $i = 0$, $\phi_{2}$ needs to be satisfied now, at time point $i$, for the query to be satisfied.
            There are no past time points $k$, $\mathsf{max}((i-1)-(p-1),0) \leq k \leq 0-1$.
            Since $i = 0$ is true, $\mathfrak{I}, i \models \mathfrak{a}_{\phi_{2}}(\phi_{2})$ is equivalent to there is $k$, $\mathsf{max}(i-p,0) \leq k \leq 0$, with $\mathfrak{I}, k \models \mathfrak{a}_{\phi_{2}}(\phi_{2})$
            and $\mathfrak{I}, j \models \mathfrak{a}_{\phi_{1}}(\phi_{1})$ for all $j, k < j \leq 0$, and $i > 0$ is not true, thus the ``or''-statement does not affect satisfaction.
        \end{itemize}
    \end{enumerate}
\end{proof}

\lemmaMTOsPhiZero*
\begin{proof}
    It is shown by induction on the structure of the subqueries $\psi \in \mathsf{Sub}(\phi)$ that $\mathsf{eval}^{n}(\Phi_{0}(\psi))$
    is equal to $\mathsf{Ans}(\psi, \mathfrak{I}^{(n)},0)$ for all $n \geq 0$.
    The missing cases can be found in Appendix \ref{ch:appendixC}.\\
    If $\psi = \Circle_{0} \psi_{1}$, $\psi = \CIRCLE_{0} \psi_{1}$, $\psi = \Circle^{-}_{0} \psi_{1}$, $\psi = \CIRCLE^{-}_{0} \psi_{1}$, $\psi = \Box_{0} \psi_{1}$, $\psi = \Box^{-}_{0} \psi_{1}$, $\psi = \Diamond_{0} \psi_{1}$ or $\psi = \Diamond^{-}_{0} \psi_{1}$, then
    \[\mathsf{eval}^{n}(\Phi_{0}(\psi)) = \mathsf{eval}^{n}(\Phi_{0}(\psi_{1})).\]
    This is by induction equal to $Ans(\psi_{1}, \mathfrak{I}^{(n)},0)$ which then is, as shown in Proposition \ref{prop:extension/metrictemporaloperators/prop2},
    equal to $\mathsf{Ans}(\psi, \mathfrak{I}^{(n)},0)$.\\
    If $\psi = \psi_{1} \mathsf{U}_{0} \psi_{2}$ or $\psi = \psi_{1} \mathsf{S}_{0} \psi_{2}$, then
    \[\mathsf{eval}^{n}(\Phi_{0}(\psi)) = \mathsf{eval}^{n}(\Phi_{0}(\psi_{2})).\]
    This is by induction equal to $Ans(\psi_{2}, \mathfrak{I}^{(n)},0)$ which then is, as shown in Proposition \ref{prop:extension/metrictemporaloperators/prop2},
    equal to $\mathsf{Ans}(\psi, \mathfrak{I}^{(n)},0)$.\\
    If $\psi = \Circle^{-}_{p} \psi_{1}$, $\psi = \CIRCLE^{-}_{p} \psi_{1}$, $\psi = \Box^{-}_{p} \psi_{1}$ or $\psi = \Diamond^{-}_{p} \psi_{1}$ and $p > 0$, then
    \[\mathsf{eval}^{n}(\Phi_{0}(\psi)) = \mathsf{eval}^{n}(\Phi_{0}(\psi_{1})).\]
    This is by induction equal to $\mathsf{Ans}(\psi_{1}, \mathfrak{I}^{(n)},0)$ which then is, as shown in Proposition \ref{prop:extension/metrictemporaloperators/prop3},
    equal to $\mathsf{Ans}(\psi, \mathfrak{I}^{(n)},0)$.\\
    If $\psi = \Circle_{p} \psi_{1}$ and $p > 0$, then
    \begin{equation}
        \notag
        \label{eq:extension/metrictemporaloperators/proofLemma5/eq1App}
        \begin{split}
            \mathsf{eval}^{n}(\Phi_{0}(\psi)) &= \mathsf{eval}^{n}(x^{\Circle_{p} \psi_{1}}_{0}) \\
            &= \left\{\begin{array}{lr}
                          \mathsf{Ans}(\Circle_{p-1} \psi_{1}, \mathfrak{I}^{(n)},1) & \text{if } n > 0\\
                          \emptyset      & \text{if } n = 0\\
            \end{array}\right\} \\
            &= \mathsf{Ans}(\psi, \mathfrak{I}^{(n)},0)
        \end{split}
    \end{equation}
    If $\psi = \CIRCLE_{p} \psi_{1}$ and $p > 0$, then
    \begin{equation}
        \notag
        \label{eq:extension/metrictemporaloperators/proofLemma5/eq2App}
        \begin{split}
            \mathsf{eval}^{n}(\Phi_{0}(\psi)) &= \mathsf{eval}^{n}(x^{\CIRCLE_{p} \psi_{1}}_{0}) \\
            &= \left\{\begin{array}{lr}
                          \mathsf{Ans}(\CIRCLE_{p-1} \psi_{1}, \mathfrak{I}^{(n)},1) & \text{if } n > 0\\
                          \Delta^{\mathsf{N}_\mathsf{V}}      & \text{if } n = 0\\
            \end{array}\right\} \\
            &= \mathsf{Ans}(\psi, \mathfrak{I}^{(n)},0)
        \end{split}
    \end{equation}
    If $\psi = \Box_{p} \psi_{1}$ and $p > 0$, then
    \begin{equation}
        \notag
        \label{eq:extension/metrictemporaloperators/proofLemma5/eq3App}
        \begin{split}
            \mathsf{eval}^{n}(\Phi_{0}(\psi)) &= \mathsf{eval}^{n}(\Phi_{0}(\psi_{1})) \cap \mathsf{eval}^{n}(x^{\Box_{p} \psi_{1}}_{0}) \\
            &= \mathsf{Ans}(\psi_{1}, \mathfrak{I}^{(n)},0) \cap \left\{\begin{array}{lr}
                                                                            \mathsf{Ans}(\Box_{p-1} \psi_{1}, \mathfrak{I}^{(n)},1) & \text{if } n > 0\\
                                                                            \Delta^{\mathsf{N}_\mathsf{V}}       & \text{if } n = 0\\
            \end{array}\right\} \\
            &= \mathsf{Ans}(\psi, \mathfrak{I}^{(n)},0)
        \end{split}
    \end{equation}
    If $\psi = \Diamond_{p} \psi_{1}$ and $p > 0$, then
    \begin{equation}
        \notag
        \label{eq:extension/metrictemporaloperators/proofLemma5/eq4App}
        \begin{split}
            \mathsf{eval}^{n}(\Phi_{0}(\psi)) &= \mathsf{eval}^{n}(\Phi_{0}(\psi_{1})) \cup \mathsf{eval}^{n}(x^{\Diamond_{p} \psi_{1}}_{0}) \\
            &= \mathsf{Ans}(\psi_{1}, \mathfrak{I}^{(n)},0) \cup \left\{\begin{array}{lr}
                                                                            \mathsf{Ans}(\Diamond_{p-1} \psi_{1}, \mathfrak{I}^{(n)},1) & \text{if } n > 0\\
                                                                            \emptyset            & \text{if } n = 0\\
            \end{array}\right\} \\
            &= \mathsf{Ans}(\psi, \mathfrak{I}^{(n)},0)
        \end{split}
    \end{equation}
    If $\psi = \psi_{1} \mathsf{U}_{p} \psi_{2}$ and $p > 0$, then
    \begin{equation}
        \notag
        \label{eq:extension/metrictemporaloperators/proofLemma5/eq5App}
        \begin{split}
            \mathsf{eval}^{n}(\Phi_{0}(\psi)) &= \mathsf{eval}^{n}(\Phi_{0}(\psi_{2})) \cup (\mathsf{eval}^{n}(\Phi_{0}(\psi_{1})) \cap \mathsf{eval}^{n}(x^{\psi_{1} \mathsf{U}_{p} \psi_{2}}_{0})) \\
            &= \mathsf{Ans}(\psi_{2}, \mathfrak{I}^{(n)},0)\cup(\mathsf{Ans}(\psi_{1}, \mathfrak{I}^{(n)},0) \cap \left\{\begin{array}{lr}
                                                                                                                             \mathsf{Ans}(\psi_{1} \mathsf{U}_{p-1} \psi_{2}, \mathfrak{I}^{(n)},1) & \text{if } n > 0\\
                                                                                                                             \emptyset       & \text{if } n = 0\\
            \end{array}\right\}) \\
            &= \mathsf{Ans}(\psi, \mathfrak{I}^{(n)},0)
        \end{split}
    \end{equation}
    If $\psi = \psi_{1} \mathsf{S}_{p} \psi_{2}$ and $p > 0$, then
    \begin{equation}
        \notag
        \label{eq:extension/metrictemporaloperators/proofLemma5/eq6App}
        \begin{split}
            \mathsf{eval}^{n}(\Phi_{0}(\psi)) &= \mathsf{eval}^{n}(\Phi_{0}(\psi_{2})) \\
            &= \mathsf{Ans}(\psi_{2}, \mathfrak{I}^{(n)},0) \\
            &= \mathsf{Ans}(\psi, \mathfrak{I}^{(n)},0)
        \end{split}
    \end{equation}
\end{proof}

\lemmaMTOsPhiiminus*
\begin{proof}
    It is shown by induction on the structure of the subqueries $\psi \in \mathsf{Sub}(\phi)$ that $\mathsf{eval}^{n}(\Phi^{0}_{i}(\psi))$
    is equal to $\mathsf{Ans}(\psi, \mathfrak{I}^{(n)},i)$ for all $n \geq i$. \\

    If $\psi = \Circle_{0} \psi_{1}$, $\psi = \CIRCLE_{0} \psi_{1}$, $\psi = \Circle^{-}_{0} \psi_{1}$, $\psi = \CIRCLE^{-}_{0} \psi_{1}$, $\psi = \Box_{0} \psi_{1}$, $\psi = \Box^{-}_{0} \psi_{1}$, $\psi = \Diamond_{0} \psi_{1}$ or $\psi = \Diamond^{-}_{0} \psi_{1}$, then
    \begin{equation}
        \notag
        \label{eq:extension/metrictemporaloperators/proofLemma6/eq1App}
        \begin{split}
            \mathsf{eval}^{n}(\Phi^{0}_{i}(\psi)) &= \mathsf{eval}^{n}(\Phi^{0}_{i}(\psi_{1})) \\
            &= \mathsf{Ans}(\psi_{1}, \mathfrak{I}^{(n)},i) \\
            &= \mathsf{Ans}(\psi, \mathfrak{I}^{(n)},i)
        \end{split}
    \end{equation}
    If $\psi = \psi_{1} \mathsf{U}_{0} \psi_{2}$ or $\psi = \psi_{1} \mathsf{S}_{0} \psi_{2}$, then
    \begin{equation}
        \notag
        \label{eq:extension/metrictemporaloperators/proofLemma6/eq2App}
        \begin{split}
            \mathsf{eval}^{n}(\Phi^{0}_{i}(\psi)) &= \mathsf{eval}^{n}(\Phi^{0}_{i}(\psi_{2})) \\
            &= \mathsf{Ans}(\psi_{2}, \mathfrak{I}^{(n)},i) \\
            &= \mathsf{Ans}(\psi, \mathfrak{I}^{(n)},i)
        \end{split}
    \end{equation}
    If $\psi = \Circle_{p} \psi_{1}$ and $p > 0$, then
    \begin{equation}
        \notag
        \label{eq:extension/metrictemporaloperators/proofLemma6/eq3App}
        \begin{split}
            \mathsf{eval}^{n}(\Phi^{0}_{i}(\psi)) &= \mathsf{eval}^{n}(x^{\Circle_{p} \psi_{1}}_{i}) \\
            &= \left\{\begin{array}{lr}
                          \mathsf{Ans}(\Circle_{p-1} \psi_{1}, \mathfrak{I}^{(n)},i+1) & \text{if } n > i\\
                          \emptyset      & \text{if } n = i\\
            \end{array}\right\} \\
            &= \mathsf{Ans}(\psi, \mathfrak{I}^{(n)},i)
        \end{split}
    \end{equation}
    If $\psi = \CIRCLE_{p} \psi_{1}$ and $p > 0$, then
    \begin{equation}
        \notag
        \label{eq:extension/metrictemporaloperators/proofLemma6/eq4App}
        \begin{split}
            \mathsf{eval}^{n}(\Phi^{0}_{i}(\psi)) &= \mathsf{eval}^{n}(x^{\CIRCLE_{p} \psi_{1}}_{i}) \\
            &= \left\{\begin{array}{lr}
                          \mathsf{Ans}(\CIRCLE_{p-1} \psi_{1}, \mathfrak{I}^{(n)},i+1) & \text{if } n > i\\
                          \Delta^{\mathsf{N}_\mathsf{V}}      & \text{if } n = i\\
            \end{array}\right\} \\
            &= \mathsf{Ans}(\psi, \mathfrak{I}^{(n)},i)
        \end{split}
    \end{equation}
    If $\psi = \Circle^{-}_{p} \psi_{1}$ and $p > 0$, then
    \begin{equation}
        \notag
        \label{eq:extension/metrictemporaloperators/proofLemma6/eq5App}
        \begin{split}
            \mathsf{eval}^{n}(\Phi^{0}_{i}(\psi)) &= \mathsf{eval}^{n}(\Phi_{i-1}(\Circle^{-}_{p-1} \psi_{1})) \\
            &= \left\{\begin{array}{lr}
                          \mathsf{Ans}(\Circle^{-}_{p-1} \psi_{1}, \mathfrak{I}^{(n)},i-1) & \text{if } i > 0\\
                          \emptyset      & \text{if } i = 0\\
            \end{array}\right\} \\
            &= \mathsf{Ans}(\psi, \mathfrak{I}^{(n)},i)
        \end{split}
    \end{equation}
    If $\psi = \CIRCLE^{-}_{p} \psi_{1}$ and $p > 0$, then
    \begin{equation}
        \notag
        \label{eq:extension/metrictemporaloperators/proofLemma6/eq6App}
        \begin{split}
            \mathsf{eval}^{n}(\Phi^{0}_{i}(\psi)) &= \mathsf{eval}^{n}(\Phi_{i-1}(\CIRCLE^{-}_{p-1} \psi_{1})) \\
            &= \left\{\begin{array}{lr}
                          \mathsf{Ans}(\CIRCLE^{-}_{p-1} \psi_{1}, \mathfrak{I}^{(n)},i-1) & \text{if } i > 0\\
                          \Delta^{\mathsf{N}_\mathsf{V}}      & \text{if } i = 0\\
            \end{array}\right\} \\
            &= \mathsf{Ans}(\psi, \mathfrak{I}^{(n)},i)
        \end{split}
    \end{equation}
    If $\psi = \Box_{p} \psi_{1}$ and $p > 0$, then
    \begin{equation}
        \notag
        \label{eq:extension/metrictemporaloperators/proofLemma6/eq7App}
        \begin{split}
            \mathsf{eval}^{n}(\Phi^{0}_{i}(\psi)) &= \mathsf{eval}^{n}(\Phi^{0}_{i}(\psi_{1})) \cap \mathsf{eval}^{n}(x^{\Box_{p} \psi_{1}}_{i}) \\
            &= \mathsf{Ans}(\psi_{1}, \mathfrak{I}^{(n)},i) \cap \left\{\begin{array}{lr}
                                                                            \mathsf{Ans}(\Box_{p-1} \psi_{1}, \mathfrak{I}^{(n)},i+1) & \text{if } n > i\\
                                                                            \Delta^{\mathsf{N}_\mathsf{V}}       & \text{if } n = i\\
            \end{array}\right\} \\
            &= \mathsf{Ans}(\psi, \mathfrak{I}^{(n)},i)
        \end{split}
    \end{equation}
    If $\psi = \Box^{-}_{p} \psi_{1}$ and $p > 0$, then
    \begin{equation}
        \notag
        \label{eq:extension/metrictemporaloperators/proofLemma6/eq8App}
        \begin{split}
            \mathsf{eval}^{n}(\Phi^{0}_{i}(\psi)) &= \mathsf{eval}^{n}(\Phi^{0}_{i}(\psi_{1})) \cap \mathsf{eval}^{n}(\Phi_{i-1}(\Box^{-}_{p} \psi_{1})) \\
            &= \mathsf{Ans}(\psi_{1}, \mathfrak{I}^{(n)},i) \cap \mathsf{Ans}(\Box^{-}_{p-1} \psi_{1}, \mathfrak{I}^{(n)},i-1) \\
            &= \mathsf{Ans}(\psi, \mathfrak{I}^{(n)},i)
        \end{split}
    \end{equation}
    If $\psi = \Diamond_{p} \psi_{1}$ and $p > 0$, then
    \begin{equation}
        \notag
        \label{eq:extension/metrictemporaloperators/proofLemma6/eq9App}
        \begin{split}
            \mathsf{eval}^{n}(\Phi^{0}_{i}(\psi)) &= \mathsf{eval}^{n}(\Phi^{0}_{i}(\psi_{1})) \cup \mathsf{eval}^{n}(x^{\Diamond_{p-1} \psi_{1}}_{i}) \\
            &= \mathsf{Ans}(\psi_{1}, \mathfrak{I}^{(n)},i) \cup \left\{\begin{array}{lr}
                                                                            \mathsf{Ans}(\Diamond_{p-1} \psi_{1}, \mathfrak{I}^{(n)},i+1) & \text{if } n > i\\
                                                                            \emptyset            & \text{if } n = i\\
            \end{array}\right\} \\
            &= \mathsf{Ans}(\psi, \mathfrak{I}^{(n)},i)
        \end{split}
    \end{equation}
    If $\psi = \Diamond^{-}_{p} \psi_{1}$ and $p > 0$, then
    \begin{equation}
        \notag
        \label{eq:extension/metrictemporaloperators/proofLemma6/eq10App}
        \begin{split}
            \mathsf{eval}^{n}(\Phi^{0}_{i}(\psi)) &= \mathsf{eval}^{n}(\Phi^{0}_{i}(\psi_{1})) \cup \mathsf{eval}^{n}(\Phi_{i-1}(\Box^{-}_{p-1} \psi_{1})) \\
            &= \mathsf{Ans}(\psi_{1}, \mathfrak{I}^{(n)},i) \cup \mathsf{Ans}(\Box^{-}_{p-1} \psi_{1}, \mathfrak{I}^{(n)},i-1) \\
            &= \mathsf{Ans}(\psi, \mathfrak{I}^{(n)},i)
        \end{split}
    \end{equation}
    If $\psi = \psi_{1} \mathsf{U}_{p} \psi_{2}$ and $p > 0$, then
    \begin{equation}
        \notag
        \label{eq:extension/metrictemporaloperators/proofLemma6/eq11App}
        \begin{split}
            \mathsf{eval}^{n}(\Phi^{0}_{i}(\psi)) &= \mathsf{eval}^{n}(\Phi^{0}_{i}(\psi_{2})) \cup (\mathsf{eval}^{n}(\Phi^{0}_{i}(\psi_{1})) \cap \mathsf{eval}^{n}(x^{\psi_{1} \mathsf{U}_{p} \psi_{2}}_{i})) \\
            &= \mathsf{Ans}(\psi_{2}, \mathfrak{I}^{(n)},i)\cup(\mathsf{Ans}(\psi_{1}, \mathfrak{I}^{(n)},i) \cap \left\{\begin{array}{lr}
                                                                                                                             \mathsf{Ans}(\psi_{1} \mathsf{U}_{p-1} \psi_{2}, \mathfrak{I}^{(n)},i+1) & \text{if } n > i\\
                                                                                                                             \emptyset       & \text{if } n = i\\
            \end{array}\right\}) \\
            &= \mathsf{Ans}(\psi, \mathfrak{I}^{(n)},i)
        \end{split}
    \end{equation}
    If $\psi = \psi_{1} \mathsf{S}_{p} \psi_{2}$ and $p > 0$, then
    \begin{equation}
        \notag
        \label{eq:extension/metrictemporaloperators/proofLemma6/eq12App}
        \begin{split}
            \mathsf{eval}^{n}(\Phi^{0}_{i}(\psi)) &= \mathsf{eval}^{n}(\Phi^{0}_{i}(\psi_{2})) \cup (\mathsf{eval}^{n}(\Phi^{0}_{i}(\psi_{1})) \cap \Phi_{i-1}(\psi_{1} \mathsf{S}_{p-1} \psi_{2})) \\
            &= \mathsf{Ans}(\psi_{2}, \mathfrak{I}^{(n)},i)\cup(\mathsf{Ans}(\psi_{1}, \mathfrak{I}^{(n)},i) \cap \mathsf{Ans}(\psi_{1} \mathsf{S}_{p-1} \psi_{2}, \mathfrak{I}^{(n)},i-1)) \\
            &= \mathsf{Ans}(\psi, \mathfrak{I}^{(n)},i)
        \end{split}
    \end{equation}
\end{proof}